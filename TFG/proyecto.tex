\documentclass[a4paper,11pt]{article}
%\documentclass[a4paper,twoside,11pt,titlepage]{book}
\usepackage{listings}
\usepackage[utf8]{inputenc}
\usepackage[spanish]{babel}
\usepackage[backend=biber, style=alphabetic]{biblatex} 
\usepackage{amsmath}
\usepackage{amssymb}
\usepackage{algorithm}
\usepackage{algpseudocode}
\usepackage{multirow}
\usepackage{subfig}
\usepackage{verbatim}

\addbibresource{bibliografia.bib}
%Que no se muestren las imagenes
%\setkeys{Gin}{draft}

% \usepackage[style=list, number=none]{glossary} %
%\usepackage{titlesec}
%\usepackage{pailatino}

\decimalpoint
\usepackage{dcolumn}
\newcolumntype{.}{D{.}{\esperiod}{-1}}
\makeatletter
\addto\shorthandsspanish{\let\esperiod\es@period@code}
\makeatother


%\usepackage[chapter]{algorithm}
\RequirePackage{verbatim}
%\RequirePackage[Glenn]{fncychap}
\usepackage{fancyhdr}
\usepackage{graphicx}
\usepackage{afterpage}



\usepackage{longtable}

\usepackage[pdfborder={000}]{hyperref} %referencia


%Math packages
\usepackage{dsfont}

% ********************************************************************
% Re-usable information
% ********************************************************************
\newcommand{\myTitle}{TFG\xspace}
\newcommand{\myDegree}{Grado en ...\xspace}
\newcommand{\myName}{Eduardo Morales Muñoz\xspace}
\newcommand{\myProf}{Pablo Mesejo Santiago\xspace}
\newcommand{\myOtherProf}{Javier Merí de la Maza\xspace}
%\newcommand{\mySupervisor}{Put name here\xspace}
\newcommand{\myFaculty}{Escuela Técnica Superior de Ingenierías Informática y de
Telecomunicación\xspace}
\newcommand{\myFacultyShort}{E.T.S. de Ingenierías Informática y de
Telecomunicación\xspace}
\newcommand{\myDepartment}{Departamento de ...\xspace}
\newcommand{\myUni}{\protect{Universidad de Granada}\xspace}
\newcommand{\myLocation}{Granada\xspace}
\newcommand{\myTime}{\today\xspace}
\newcommand{\myVersion}{Version 0.1\xspace}



\hypersetup{
pdfauthor = {\myName eedduuy@correo.ugr.es},
pdftitle = {\myTitle TFG},
pdfsubject = {},
pdfkeywords = {palabra_clave1, palabra_clave2, palabra_clave3, ...},
pdfcreator = {LaTeX con el paquete ....},
pdfproducer = {pdflatex}
}

%\hyphenation{}


%\usepackage{doxygen/doxygen}
%\usepackage{pdfpages}
\usepackage{url}
\usepackage{colortbl,longtable}
\usepackage[stable]{footmisc}
%\usepackage{index}

\makeindex
%\usepackage[style=long, cols=2,border=plain,toc=true,number=none]{glossary}
% \makeglossary

% Definición de comandos que me son tiles:
%\renewcommand{\indexname}{Índice alfabético}
%\renewcommand{\glossaryname}{Glosario}

\pagestyle{fancy}
\fancyhf{}
\fancyhead[LO]{\leftmark}
\fancyhead[RE]{\rightmark}
\fancyhead[RO,LE]{\textbf{\thepage}}
%\renewcommand{\chaptermark}[1]{\markboth{\textbf{#1}}{}}
%\renewcommand{\sectionmark}[1]{\markright{\textbf{\thesection. #1}}}

\setlength{\headheight}{1.5\headheight}

\newcommand{\HRule}{\rule{\linewidth}{0.5mm}}
%Definimos los tipos teorema, ejemplo y definición podremos usar estos tipos
%simplemente poniendo \begin{teorema} \end{teorema} ...
\newtheorem{teorema}{Teorema}[section]
\newtheorem{ejemplo}{Ejemplo}[section]
\newtheorem{definicion}{Definición}[section]
\newtheorem{proposicion}{Proposición}[section]



\definecolor{gray97}{gray}{.97}
\definecolor{gray75}{gray}{.75}
\definecolor{gray45}{gray}{.45}
\definecolor{gray30}{gray}{.94}

\lstset{ frame=Ltb,
     framerule=0.5pt,
     aboveskip=0.5cm,
     framextopmargin=3pt,
     framexbottommargin=3pt,
     framexleftmargin=0.1cm,
     framesep=0pt,
     rulesep=.4pt,
     backgroundcolor=\color{gray97},
     rulesepcolor=\color{black},
     %
     stringstyle=\ttfamily,
     showstringspaces = false,
     basicstyle=\scriptsize\ttfamily,
     commentstyle=\color{gray45},
     keywordstyle=\bfseries,
     %
     numbers=left,
     numbersep=6pt,
     numberstyle=\tiny,
     numberfirstline = false,
     breaklines=true,
   }
 
% minimizar fragmentado de listados
\lstnewenvironment{listing}[1][]
   {\lstset{#1}\pagebreak[0]}{\pagebreak[0]}

\newcommand{\bigrule}{\titlerule[0.5mm]}


%Para conseguir que en las páginas en blanco no ponga cabecerass
\makeatletter
\def\clearpage{%
  \ifvmode
    \ifnum \@dbltopnum =\m@ne
      \ifdim \pagetotal <\topskip
        \hbox{}
      \fi
    \fi
  \fi
  \newpage
  \thispagestyle{empty}
  \write\m@ne{}
  \vbox{}
  \penalty -\@Mi
}
\makeatother

\pagenumbering{roman}
\usepackage{pdfpages}
\begin{document}
\begin{titlepage}
 
 
\newlength{\centeroffset}
\setlength{\centeroffset}{-0.5\oddsidemargin}
\addtolength{\centeroffset}{0.5\evensidemargin}
\thispagestyle{empty}

\noindent\hspace*{\centeroffset}\begin{minipage}{\textwidth}

\centering
\includegraphics[width=0.7\textwidth]{Plantilla_TFG_latex/imagenes/logo_ugr.jpg}\\[1.4cm]

\textsc{ \Large TRABAJO FIN DE GRADO\\[0.2cm]}
\textsc{ Doble Grado Ingeniería Informática y Matemáticas}\\[1cm]
% Upper part of the page
% 
% Title
{\huge\bfseries Análisis del algoritmo de gradiente descendente y estudio empírico comparativo con técnicas metaheurísticas\\
}
\noindent\rule[-1ex]{\textwidth}{3pt}\\[3.5ex]

\end{minipage}

\vspace{1cm}
\noindent\hspace*{\centeroffset}\begin{minipage}{\textwidth}
\centering

\textbf{Autor}\\ {Eduardo Morales Muñoz}\\[2.5ex]
\textbf{Directores}\\
{Pablo Mesejo Santiago\\
Javier Merí de la Maza}\\[1cm]
\raggedright
\hspace{1.7cm}
\includegraphics[width=0.3\textwidth]{Plantilla_TFG_latex/imagenes/etsiit_logo.png} \hspace{2cm}
\includegraphics[width=0.12\textwidth]{Plantilla_TFG_latex/imagenes/fcienciasLogo.png}\\[0.1cm]
\centering
\textsc{Facultad de Ciencias \\ Escuela Técnica Superior de Ingenierías Informática y de Telecomunicación}\\
\textsc{---}\\
Granada, diciembre de 2024
\end{minipage}
%\addtolength{\textwidth}{\centeroffset}
%\vspace{\stretch{2}}
\end{titlepage}



%\input{prefacios/prefacio}
%\frontmatter
\tableofcontents
\newpage
%\listoffigures
%\listoftables
%
%\mainmatter
%\setlength{\parskip}{5pt}
\pagenumbering{arabic}
\setcounter{page}{1}
\setcounter{section}{0}
%\input{Plantilla_TFG_latex/Informatica/capitulos/01_Introduccion}


\newpage

%\part{Parte matemática: gradiente descendente y \textit{backpropagation}}
\vspace{4cm}

\newpage 


\section{Introducción}

El aprendizaje automático es una rama de la inteligencia artifical en la que los sistemas son capaces de adquirir conocimiento a partir de datos sin procesar \cite{GoodFellowBook}. Se dice que un programa aprende de la experiencia $E$ respecto de alguna tarea $T$ y una medición de rendimiento $P$ si su rendimiento en $T$, medido por $P$, mejora con la experiencia $E$ \cite{mitchell1997machine}. Nos referimos a este programa como modelo. Existen muchos tipos o subramas de aprendizaje automático dependiendo de la naturaleza de esta tarea $T$ y de su medidor de rendimiento $P$. 

El entrenamiento de un modelo es el proceso de optimizar sus parámetros (equivalentemente pesos), es decir, su representación interna; para minimizar una función de coste (equivalentemente función de error o de pérdida) $C$ que mide el error en el rendimiento. El dominio de dicha función es el espacio de valores que pueden tomar los pesos, normalmente representado de forma tensorial; y su imagen es comúnmente un real no negativo. El objetivo principal del entrenamiento es que el modelo sea capaz de aprender los patrones en un conjunto de datos para luego poder generalizarlos en otros que no ha visto previamente. Diremos que existe un sobreajuste cuando se aprenden los patrones específicos de los datos pero luego no se generaliza bien. La estrategia que usamos para optimizar los pesos es llamada algoritmo de aprendizaje.

El aprendizaje profundo es un paradigma del aprendizaje automático en el que los modelos tienen varios niveles de representación obtenidos a través de la composición de módulos sencillos pero comúnmente no lineales, que transforman la representación de los datos sin procesar hacia un nivel de abstracción mayor \cite{lecun2015deep}. Esta rama comenzó a ganar peso en la década de los 2000s y un punto de inflexión fue el resultado de la competición de ImageNet \footnote{http://www.image-net.org/challenges/LSVRC/} en 2012 \cite{NIPS2012_c399862d}. Actualmente este enfoque es el que mejores resultados consigue, siendo una parte fundamental en la investigación y estructura de las grandes compañías tecnológicas y pudiendo ofrecer aplicaciones comerciales a nivel usuario \cite{Sejnowski18, lecunnDeepForAI}.

La mayoría de los modelos en aprendizaje automático se entrenan usando técnicas basadas en el algoritmo de aprendizaje de gradiente descendente (equivalentemente descenso del gradiente), ya que es la estrategia que mejores resultados ofrece actualmente en cuanto a capacidad de generalización del modelo y rendimiento computacional \cite{GoodFellowBook, CauchyGD}. Ésta se basa en la idea de que puedo moverme hacia puntos de menor valor en la función de error del modelo realizando pequeños movimientos en  sentido contrario a su gradiente como se esquematiza en la figura \ref{fig:1.GD}, con el objetivo de minimizar el valor de salida. Al tratarse de un algoritmo iterativo, es fundamental estudiar su convergencia, que depende de varios factores y se enfrenta a diversas dificultades, como veremos en secciones posteriores.

El algoritmo de \textit{backpropagation} (BP) permite transmitir la información desde la salida de la función de coste hacia atrás en un modelo con varios módulos de abstracción para así poder computar el gradiente de una manera sencilla y eficiente \cite{rumelbackprop}. Aunque existen otras posibilidades a la hora de realizar éste cómputo, BP es la más usada y extendida gracias a propiedades como su flexibilidad, eficiencia y escalabilidad, que lo hacen destacar por encima de otras opciones \cite{GoodFellowBook}. 

\begin{figure}
    \centering
    \includegraphics[width=0.75\linewidth]{Plantilla_TFG_latex//imagenes//Mat//1.intro/1.1GDMatIntroGoodFellowBook.png}
    \caption{Esquematización de la estrategia de descenso del gradiente en un modelo con un solo parámetro x. El eje horizontal representa los valores que toma éste y el vertical representa el error del modelo en función de x. Imagen obtenida y traducida del libro \cite{GoodFellowBook}}
    \label{fig:1.GD}
\end{figure}



Dependiendo de la familia de modelos que usemos podremos utilizar una estrategia de aprendizaje distinta, como el caso del \textit{Perceptron} y su \textit{Perceptron Learning Algorithm} \cite{patternrecog}. En otros casos como la regresión lineal se usa la estrategia de descenso de gradiente pero el gradiente no tiene por qué calcularse a través de BP. Esto se debe a que en este caso se puede obtener eficientemente a través de librerías matemáticas como \textit{numpy} \footnote{https://numpy.org/} en el caso del lenguaje \textit{python} \footnote{https://es.python.org/}, ya que esta familia de modelos conllevan menos costo computacional en sus cálculos principalmente debido al escaso número de parámetros en comparación con los de aprendizaje profundo. Para éste sí que es necesario el uso de BP en el caso de que elijamos entrenar mediante gradiente descendente, ya que aunque existen otras alternativas como los métodos numéricos o algunas aproximaciones recientes , no consiguen igualar su rendimiento \cite{EffBackProp, GoodFellowBook, alternativabacknumerical, alternativabackprop1}.


Otra de las características de este algoritmo para el cálculo del gradiente es que los conceptos en los que se basa son simples: optimización, diferenciación, derivadas parciales y regla de la cadena. Lo cual lo convierte a priori en objeto de estudio accesible. En la práctica, los cálculos que se realizan en esta estrategia se implementan a través de la diferenciación automática, que es una técnica más general que extiende a BP y se usa para el cómputo de derivadas de funciones numéricas de una manera eficiente y precisa \cite{AutomaticDiff}.% Ésta aprovecha el hecho de que cada cálculo que se realiza en un ordenador queda reducido a una secuencia de operaciones aritméticas elementales (suma, resta, multiplicación...) y funciones elementales (exponencial, trigonométricas, ...) para aplicar de forma repetida la regla de la cadena a estos elementos básicos hasta poder obtener derivadas de orden arbitrario. 



\subsection{Motivación}

Tenemos pues que el aprendizaje profundo es el paradigma del aprendizaje automático que mejores resultados obtiene actualmente y más desarrollo e investigación está concentrando, basa el entrenamiento (una de las partes fundamentales que determinan el rendimiento del modelo, además de su arquitectura) de los modelos casi por completo en el algoritmo de descenso de gradiente, ya que es el que mejores resultados de generalización ofrece. Éste a su vez depende casi enteramente del algoritmo de BP para calcular el gradiente, ya que aunque existan otras alternativas no son realmente viables. Tanto es así que es muy común la confusión entre éste algoritmo y el de gradiente descentente, que se suelen tomar por la misma cosa. Queda así clara la importancia que tiene BP en el campo del aprendizaje profundo y por extensión también al aprendizaje automático. También conviene destacar la cantidad de veces que se utiliza ésta técnica durante el entrenamiento de un modelo. Cada vez que se actualizan los pesos debemos calcular el gradiente, y teniendo en cuenta la duración de los entrenamientos de los modelos más grandes (con mayor número de parámetros) este algoritmo puede ser usado miles de veces durante un entrenamiento.

Su eficiencia, escalabilidad y flexibilidad lo han convertido en la opción por defecto para el entrenamiento basado en gradiente descendente para modelos de aprendizaje profundo, sin embargo no hay que olvidar que no se trata de una tarea sencilla: la obtención de un mínimo global y la verificación, dado un punto, de que es un mínimo global, se trata de un problema NP-Completo generalmente \cite{NPHardProblem}, por lo que se buscan estrategias aproximadas capaces de obtener buenas soluciones en tiempos razonables. Uno de los problemas abiertos en el aprendizaje profundo y en el que influye directamente BP es la reducción computacional del entrenamiento: si se ajustan los pesos en un modelo con un número muy alto de parámetros y usando un conjunto de entrenamiento muy grande (que es una tendencia reciente en aprendizaje profundo), los recursos computacionales pueden resultar insuficientes incluso para las grandes compañías, pudiendo requerir de meses para el  entrenamiento. Por lo que se necesitan algoritmos más escalables y eficientes para afrontarlo \cite{Problem3_accel}.

Por ello resulta esencial, mientras no existan alternativas viables, poder ofrecer mejoras a este algoritmo para mejorar sus cualidades. Atendiendo a la cantidad de uso y su extensión en el campo, una pequeña mejora tendría un alcance enorme. Sin embargo esta línea de investigación no es muy extensa ya que principalmente se buscan alternativas en lugar de mejoras, pudiendo deberse principalmente a que a priori puede parecer una técnica muy enrevesada y compleja. Veremos en el desarrollo de esta parte que esto no es algo cierto, y que los principios en los que se basa son muy simples. Es clave comprender su base teórica, funcionamiento e implementación práctica para poder proponer mejoras. 


\subsection{Objetivos}

El objetivo principal de esta parte es realizar una investigación sobre los algoritmos de descenso de gradiente y \textit{backpropagation}, proporcionando una visión detallada acerca de los mismos y su implementación. Para ello se divide este objetivo en varios:

\begin{enumerate}
    %\item Exposición de la importancia actual e histórica en el campo del aprendizaje automático y aprendizaje profundo, viendo su estrecha relación con el algoritmo de aprendizaje del gradiente descente.
    
    \item Definir de manera detallada la base teórica y funcionamiento del algoritmo de descenso de gradiente.

    \item Explorar su implementación a través de BP. Para ello, analizaremos su funcionamiento e implementación.

    %\item Analizar de modo teórico las principales variantes del algoritmo de descenso de gradiente. 

 %   \item Realizar una investigación y revisión teórica de las alternativas al cálculo de gradiente basadas en otras estrategias como los métodos numéricos, e inviabilidad de las mismas frente a BP.
\end{enumerate}







\section{Fundamentos previos}
A continuación se definirán los conceptos básicos necesarios con los que se trabajará durante el desarrollo de esta parte. Se tratarán los elementos necesarios que se usan en el algoritmo de gradiente descendente y BP. Se presenta únicamente el material estrictamente necesario para comprender el trabajo. Se ha usado para la elaboración de esta sección los apuntes en línea del profesor de la UGR Rafael Payá Albert en su curso de Análisis Matemático I \footnote{\url{https://www.ugr.es/~rpaya/docencia.htm\#Analisis}}.  Salvo otras especificaciones, el material de consulta para el desarrollo de esta parte matemática ha sido el curso en línea de Ciencias de Computación de la universidad Bristish Columbia \footnote{\url{https://www.cs.ubc.ca/~schmidtm/Courses/5XX-S22/}} y los libros Probabilistic Machine Learning \cite{murphy2022probabilistic} y Deep Learning \cite{GoodFellowBook}.


\subsection{Cálculo diferencial}

A continuación se definirán los principales conceptos que se usarán durante el presente TFG. Los algoritmos de gradiente descendente y BP se basan principalmente en el cálculo diferencial, y el hecho de que no usen herramientas matemáticas demasiado complejas resulta precisamente una de sus virtudes, ya que gracias a la abstracción y a un diseño ingenioso consiguen obtener grandes resultados a partir de operaciones relativamente sencillas. Empezamos con los conceptos más elementales que subyacen durante todo el trabajo.

En lo que sigue se fijan los abiertos $X \subseteq \mathbb{R}^n$, $Y \subseteq \mathbb{R}^m$ y la función $f: X \rightarrow Y$.


\begin{definicion}[Función diferenciable]
    $f$ es diferenciable en el punto $a \in X$ si existe una aplicación lineal y continua $T \in L(X,Y)$ que verifica:

    $$Df(a) = \displaystyle \lim_{x \to a} \frac{\left\| f(x)-f(a)-T(x-a)\right\|}{\left\| x-a\right\|}=0$$
    
    Decimos que $f$ es diferenciable si es diferenciable en todo punto del interior de su dominio.
\end{definicion}




\begin{definicion}[Derivada parcial en un campo escalar]
        Sea $f: X \rightarrow \mathbb{R}$. La derivada parcial de $f$ con respecto a la $k$-ésima variable $x_k$ en el punto $a = (a_1, \ldots, a_n)\in X$ se define como

	$$ \frac{\partial f}{\partial x_k}(a) = \underset{h\rightarrow0}{lim}\frac{f(a_1, \ldots, a_{k-1}, a_k + h, a_{k+1}, \ldots, a_n) - f(a_1, \ldots, a_n)}{h}$$

	si existe el límite.
\end{definicion}

Notamos por $f= \left ( f_1,f_2,\ldots, f_m \right )$ indicando las $m$ componentes de $f$ que es un campo escalar definido en $X$, siendo $f_j=\pi _j \circ f$.  En lo que sigue $x = \left ( x_1, \ldots, x_n  \right )  \in X$


\begin{definicion}[Derivada parcial]
        Sea $f: X \rightarrow Y$,  $f=(f_1, f_2, \ldots, f_m)$, $k \in I_n$. Entonces $f$ es parcialmente derivable con respecto a la $k$-ésima variable $x_k$ en $a = (a_1, \ldots, a_n)\in X$ si, y sólo si, lo es $f_j \forall j \in I_m$, en tal caso, 

        $$\frac{\partial f}{\partial x_k}(a) = \left ( \frac{\partial f_1}{\partial x_k}(a), \ldots, \frac{\partial f_m}{\partial x_k}(a) \right ) \in \mathbb{R}^m$$

        $f$ es parcialmente derivable en $a$ si, y sólo si, lo es respecto de todas sus variables.
\end{definicion}


Definimos ahora los elementos clave del proceso: el vector gradiente y la matriz jacobiana. En el algoritmo de descenso de gradiente, lo que se pretende calcular tal como indica el nombre es el vector gradiente, ya que la función de error de los modelos siempre nos devuelve un escalar, es decir que la dimensión de la imagen es 1, y la dimensión de la entrada será el número de parámetros del modelo (número de elementos que tendrá el vector gradiente). Sin embargo las matrices jacobianas también juegan un papel fundamental ya que para calcular ese vector gradiente, el algoritmo de BP necesita de cálculos intermedios, que son las matrices jacobianas asociadas entradas y salidas de las capas ocultas (que tienen mayor dimensionalidad) con respecto a parte de los parámetros (los parámetros de esa capa).

\begin{definicion}[Vector gradiente]
    Sea $f:X \rightarrow \mathbb{R}$ un campo escalar. Cuando $f$ es parcialmente derivable en $x$, el gradiente de $f$ en $x$ es el vector $\nabla f(x) \in X$ dado por 
    $$\nabla f(x) = \left ( \frac{\partial f}{\partial x_1}, \frac{\partial f}{\partial x_2}, \ldots, \frac{\partial f}{\partial x_n} \right ).$$
\end{definicion}


\begin{definicion}[Matriz jacobiana]
    Si $f$ es diferenciable en $ x \in X$, la matriz jacobiana es la matriz de la aplicación lineal $Df \in L \left ( X, Y \right )$ y se escribe como $J_f$. Viene dada por:

    $$J_f(x)= \begin{pmatrix}
 \frac{\partial f_1}{\partial x_1} & \frac{\partial f_1}{\partial x_2} & \cdots & \frac{\partial f_1}{\partial x_n} \\
 \frac{\partial f_2}{\partial x_1} & \frac{\partial f_2}{\partial x_2} & \cdots & \frac{\partial f_2}{\partial x_n} \\
 \vdots & \vdots & \ddots & \vdots \\
 \frac{\partial f_m}{\partial x_1} & \frac{\partial f_m}{\partial x_2} & \cdots & \frac{\partial f_m}{\partial x_n} \\
\end{pmatrix}= \begin{pmatrix}
 \nabla f_1(x)^T\\
 \vdots \\
 \nabla f_m(x)^T \\
\end{pmatrix}=
\begin{pmatrix}
     \frac{\partial f}{\partial x_1} \cdots \frac{\partial f}{\partial x_n}
\end{pmatrix}$$
\end{definicion}



Si f es de clase $C^2$ (derivable dos veces con sus derivadas continuas) derivamos el gradiente obtenemos una matriz cuadrada simétrica con derivadas parciales de segundo orden, a la que llamamos matriz Hessiana.

\begin{definicion}[Matriz Hessiana]
	Dado el campo escalar $f: X \rightarrow \mathbb{R}$, definimos la matriz Hessiana en el punto x como

	$$\nabla^2f(x)= \begin{pmatrix}
		\frac{\partial^2f}{\partial x^{2}_1} & \frac{\partial^2f}{\partial x_1\partial x_2} & \cdots & \frac{\partial^2f}{\partial x_1 \partial x_n}\\
		\frac{\partial^2f}{\partial x_2 \partial x_1} & \frac{\partial^2f}{\partial x^{2}_2} & \cdots & \frac{\partial^2f}{\partial x_2 \partial x_n}\\
		\vdots & \vdots & \ddots & \vdots \\
		\frac{\partial^2f}{\partial x_n \partial x_1} & \frac{\partial^2f}{\partial x_n \partial x_1} & \cdots & \frac{\partial^2f}{\partial x^{2}_n}\\
	\end{pmatrix}$$

\end{definicion}

Este es un concepto muy importante del cálculo multivariable y la optimización. Cuando hablemos del gradiente descendente, especialmente de la convergencia, usaremos algunas de sus propiedades, como por ejemplo la aproximación cuadrática para desplazamientos pequeños: $f(x + \Delta x) \approx f(x) + \nabla f(x)^T \Delta x + \frac{1}{2} \Delta ^T \nabla^2 f(x) \Delta x$. HAY UN RESULTADO QUE F ES CONVEXA SII LA MATRIZ HESSIANA ES SEMIDEFINIDA POSITIVA PERO NO SE SI PONERLO. ESTO SE APLICA A NIVEL LOCAL TAMBIEN.
		


Se presenta a continuación una de las reglas más útiles para el cálculo de diferenciales, que afirma que la composición de aplicaciones preserva la diferenciabilidad. Será parte clave en el desarrollo próximo ya que a los modelos de aprendizaje automático basados en capas podemos describirlos como una función que se descompone en una función por cada capa, por tanto será una herramienta que usaremos continuamente para calcular estas matrices jacobianas y gradientes.

\begin{teorema}[Regla de la cadena]
    Sea $Z \subseteq \mathbb{R}^p$ un abierto y sean las funciones $f:X \rightarrow Y$ y $g:Y \rightarrow Z$. Entonces si $f$ es diferenciable en $a \in X$ y g es diferenciable en $b=f(a)$ se tiene que $g \circ f$ es diferenciable en $a$ con

    $$D(g \circ f)(a) = Dg(b) \circ Df(a) = Dg(f(a)) \circ f(a)$$

    \raggedright{Si $f \in D(X, Y)$ y  $g \in D(Y,Z)$, entonces $g \circ f \in D(X, Z)$.}
    
\end{teorema}


En ocasiones en algunos modelos tenemos que lidiar con funciones que no son diferenciables en un punto, y para poder manejarlas extenderemos el concepto de diferenciabilidad a lo que llamaremos subdiferenciabilidad. Esto se expondrá más adelante ya que son conceptos que no se han explorado a lo largo del grado de matemáticas.


El último concepto, que también resulta de gran importancia en los resultados teóricos sobre la convergencia del gradiente descendente, es el de la condición de lipschitz, en concreto aplicada al gradiente. No forma parte del cálculo diferencial ya que la condición de Lipschitz no requiere diferenciabilidad, pero lo usaremos en éste ámbito ya que la condición que nos interesa usar está aplicada al gradiente.

\begin{definicion}[Función Lipschitziana]
    El campo escalar $f:X \rightarrow \mathbb{R}$ es lipschitziano si existe una constante $M \in \mathbb{R}_0^+$ que verifica:
    $$ \| f(x) - f(y) \| \leq M \| x - y \| \qquad \forall x,y \in E.$$
\end{definicion}

La definición nos dice de manera intuitiva que el gradiente de la función no puede cambiar a una velocidad arbitraria. Decimos que la función $f$ tiene gradiente lipschitziano si la condición anterior se aplica a su gradiente.

$$\| \nabla f(x) - \nabla f(y) \| \leq M \| x - y \| \qquad \forall x,y \in E .$$


La mínima constante $M_0=L$ que verifica las desigualdades anterior es denominada la constante de Lipschitz de $f$ y viene definida por 

$$L=sup \left \{ \frac{\|f(x)-f(y)\|}{\|x - y \|} : x,y \in E, x \neq y \right \}.$$



Para las funciones de clase $C^2$, es decir las que son diferenciables al menos dos veces con su derivada continua, una equivalencia a que el gradiente de $f$ sea lipschitziano es que $\nabla^2 f(x) \preceq LI \quad \forall x \in E$, esto lo usaremos luego en la demostración \ref{proof:gdconvex}. En esta expresión $L$ es la constante de Lipschitz para el gradiente de $f$ e $I$ es la matriz identidad. El símbolo $\preceq$ denota una desigualdad matricial en términos de semidefinición positiva, es decir que $LI - \nabla^2f(x)$ es una matriz semidefinida positiva. Equivalentemente para cualquier vector $z$ se tiene $z^T \left ( LI - \nabla^2f(x) \right )z \geq 0$. 


\section{Gradiente Descendente}

Se trata de un algoritmo de aprendizaje iterativo clásico, basado en el método de optimización para funciones lineales de Cauchy. Haskell Curry lo estudió por primera vez para optimización no lineal en 1944 \cite{Curry1944GDNoLin}, siendo ampliamente usado a partir de las décadas de 1950-1960. Actualmente se trata de la estrategia de entrenamiento de modelos más ampliamente usada, especialmente en los modelos de aprendizaje profundo, siendo la estrategia que mejores resultados consigue en cuanto a capacidad de generalización de los modelos y eficiencia computacional gracias a su aplicación a través del algoritmo de BP. Sin embargo a nivel práctico no se usa en su versión original, sino que a lo largo del tiempo han ido surgiendo numerosas modificaciones con el objetivo de mejorar el algoritmo en diversos ámbitos: aumento de la estabilidad y la velocidad de convergencia, reducción computacional del entrenamiento, capacidad de evitar mínimos locales, etc. Estos métodos modificados del original se conocen como optimizadores. La literatura en este sentido es extensa, es claro que el gradiente descendente sigue siendo la mejor estrategia de optimización de parámetros de un modelo de forma general \cite{MHtrainingClase}, aunque la elección del algoritmo de optimización concreto y de su ajuste depende del problema concreto que estemos tratando y generalmente se realiza de manera experimental. 

Debemos ver que el entrenamiento de los modelos, está intrínsecamente ligado a la optimización, en concreto a la minimización de la función de coste $C$. Este no es un problema sencillo, y como se ha mencionado antes se trata de un problema NP-Completo, por tanto de existir algoritmos exactos estos requieren demasiado coste computacional como para utilizarlos en la práctica, por lo que se buscan estrategias aproximadas como el descenso de gradiente para obtener buenas soluciones en un tiempo asequible. 

Otros factores a tener en cuenta son la necesidad de escapar de óptimos locales, aún no conociendo de manera explícita la función de error; y la generalización: no es  importante únicamente obtener un error bajo en el entrenamiento sino que se mantenga cuando usamos datos de entrada nuevos, ya que nuestro objetivo es ser capaces de encontrar patrones que podamos aplicar en situaciones nuevas y no ajustar el modelo a unos datos dados.

\subsection{Gradiente descendente de Cauchy}

Procedemos a describir el método original de descenso de gradiente, propuesto en 1847 por Augustin-Louis Cauchy \cite{CauchyGD}. Es una versión más primitiva y limitada que sus desarrollos posteriores pero que nos permite obtener de forma más sencilla una visión de su funcionamiento. 

Fijamos $f:\mathbb{R}^n \rightarrow \mathbb{R}_{0}^{+}$ una función continua que no toma valores negativos. Sea $x= \left ( x_1,\ldots,x_n \right ) \in \mathbb{R}^n$. Si queremos encontrar los valores de $x_1,\ldots,x_n$ que verifican $f(x)=0$, que suponemos que existen, bastará con hacer decrecer indefinidamente los valores de la función $f$ hasta que sean muy cercanos a $0$. 

Fijamos ahora unos valores concretos $x_0 \in \mathbb{R}^n$, $u=f(x_0)$,\\ $Du= \left ( D_{x_1}u, D_{x_2}u, \ldots, D_{x_n}u \right )$ y $\epsilon >0$ con $\epsilon \in \mathbb{R}^n$. Si tomamos $x_0'=x_0+\epsilon$ tendremos:
$$f(x_0')= f(x_0 + \epsilon) = u + \epsilon Du$$

Sea ahora $\eta >0$, tomando $\epsilon= - \eta Du$ con la fórmula anterior tenemos: 

$$f(x_0') = f(x_0 + \epsilon) = u - \eta \sum_{i=1}^{n}(D_{x_i}u)^2$$

Por tanto hemos obtenido un decremento en el valor de la función $f$ modificando los valores de sus variables en sentido contrario al gradiente, para $\eta$ suficientemente pequeño. El objetivo de la estrategia es repetir esta operación hasta que se desvanezca el valor de la función $f$.




\subsection{Gradiente descendente en el entrenamiento de modelos}

En el caso del entrenamiento de modelos la función que debemos minimizar es la función de coste $C$, que efectivamente es continua por ser composición de funciones continuas, como se verá más adelante. Esta función no toma valores negativos. Como no podemos realizar un cálculo continuo para comprobar con qué valores de $\eta$ la función decrece, lo hacemos de manera iterativa, y a este $\eta$ lo llamamos ratio de aprendizaje o más comúnmente \textit{learning rate}. 

Si $C(W)$ es la función de coste del modelo y $W$ representa los parámetros del modelo, entonces la regla iterativa de actualización de los pesos en la estrategia del descenso del gradiente es la siguiente:

\begin{equation}\label{eq:GD}
W_{t+1}=W_t - \eta \nabla C(W)
\end{equation}

En su descripción original, el gradiente se calcula usando todos los datos de entrenamiento, pero en versiones posteriores se propone dividir el conjunto de entrenamiento en varios subconjuntos disjuntos, denominados lotes. Cada vez que se calcula el gradiente se actualizan los pesos del modelo, y denominamos a esto una iteración. Cada vez que se usan todos los datos de entrenamiento para calcular el gradiente, ya sea tras una sola iteración usando todo el conjunto de entrenamiento o varias si dividimos en lotes, lo denominamos época.



\subsubsection{Estrategias de gradiente descendente} \label{sec:estrategias}

En base a los lotes en que dividamos el conjunto de entrenamiento tenemos varios tipos de gradiente descendente \cite{GoodFellowBook}.

\begin{itemize}
    \item \textbf{Batch Gradient Descent} (BGD): tenemos un único lote, cada iteración se corresponde con una época. Calculamos el gradiente usando todo el conjunto de entrenamiento. Esto ofrece un comportamiento mejor estudiado a nivel teórico, con más resultados demostrados; pero aumenta mucho el coste computacional del entrenamiento hasta el punto que lo vuelve demasiado lento para ser usado en la práctica.

    \item \textbf{Stochastic Gradient Descent} (SGD): Actualiza los pesos calculando el gradiente con sólo un elemento del conjunto de entrenamiento. Cada época tiene tantas iteraciones como número de elementos haya en el conjunto de entrenamiento. Esta estrategia introduce ruido en el entrenamiento ya que el gradiente se calcula de una manera aproximada, aunque esto tiene un efecto positivo ya que al provocar más irregularidad en la trayectoria de convergencia es más probable poder escapar mínimos locales. Además es más eficiente computacionalmente que el anterior y converge más rápido en la práctica.

    \item \textbf{Mini-Batch Gradient Descent} (MBGD): Se divide el conjunto de entrenamiento en $M$ lotes disjuntos de tamaño fijo, y se calcula el gradiente con cada lote, por lo que habrá $M$ iteraciones en cada época. Se consigue una aproximación del gradiente con menos error al usar más datos para su cálculo y además se siguen manteniendo las propiedades que veíamos en la anterior estrategia. Es más eficiente que la anterior al conllevar menos actualizaciones de pesos. Es prácticamente la única estrategia utilizada en la realidad ya que ofrece la mayor eficiencia computacional, estabilidad y rapidez en la convergencia.
\end{itemize}

En estos dos últimos casos, el conjunto de entrenamiento no permanece fijo, por lo que la regla de actualización de los pesos que hemos visto en \ref{eq:GD} quedaría de la siguiente manera:

\begin{equation}\label{eq:SGD}
	W_{t+1} = W_t - \eta \nabla C(X_{t+1}, W_t)
\end{equation}

Aunque la política para computar el gradiente sea distinta en estos 3 tipos, los englobaremos dentro de lo que denominaremos el algoritmo de gradiente descendente original, ya que existen varias modificaciones del algoritmo que aportan mejoras a través de modificar la regla de actualización de los pesos y no solo la cantidad de datos con la que se aproxima el gradiente.

\subsubsection{\textit{Learning rate}}

El elemento $\eta$ que observamos en la ecuación \ref{eq:GD} del gradiente descendente se denomina \textit{learning rate} y lo usamos para controlar la convergencia reduciendo el efecto de la magnitud del gradiente en la actualización de los parámetros. Este valor es positivo y situado en la práctica alrededor de 0.01 y 0.001 usualmente, aunque para su elección conviene realizar un análisis teórico previo o realizar pruebas prácticas (mucho más común) para elegir un valor adecuado. Este tipo de parámetros, que no son parte del modelo sino del algoritmo de aprendizaje, se denominan hiperparámetros. Dependiendo del tipo de algoritmo o modificación del mismo que usemos habrá diferentes hiperparámetros, siendo el \textit{learning rate} el más importante de manera general, ya que de su valor dependerá la convergencia del algoritmo, pudiendo hacer que converja demasiado lento o que directamente diverja, como podemos observar en la figura \ref{fig:lr} o en resultados sobre la convergencia en la sección \ref{sec:convergencia}.

En cuanto a la selección de los hiperparámetros, no se enfoca como un problema donde se busque el óptimo de estos valores ya que la mayoría no son tan decisivos en la convergencia como el \textit{learning rate}, y se ofrecen valores teóricos en sus papers de presentación que funcionan bien en casos generales. Si bien la convergencia es sensible a los valores iniciales de estos hiperparámetros que se tratan de optimizar a nivel experimental a través del ensayo y error, aunque no se realiza una búsqueda exhaustiva, invirtiéndose muchos más recursos computacionales en el entrenamiento.



\begin{figure}
    \centering
    \includegraphics[width=0.5\linewidth]{Plantilla_TFG_latex//imagenes//Mat//GD/lr.png}
    \caption{Visualización de cómo afecta el \textit{learning rate} según su adecuación al problema. Imagen obtenida del curso de Caltech \footnote{https://home.work.caltech.edu/slides/slides09.pdf}, tema 9 diapositiva 21}
    \label{fig:lr}
\end{figure}

Una táctica habitual es usar una política de \textit{learning rate} que decrezca conforme avanza el entrenamiento, de manera que el algoritmo avance con pasos más grandes cuando aún está lejos de un óptimo, con un objetivo explorador, y con pasos más pequeños cuando se va acercando, con un objetivo explotador, procurando una convergencia más estable. \cite{GoodFellowBook}. Otro enfoque común es tener un vector de \textit{learning rate} en lugar de un solo escalar, teniendo un valor para cada peso del modelo. 





\subsection{Subgradientes} \label{sec:subgrad}

Con el objetivo central de calcular el gradiente es lógico pensar que necesitamos ciertas condiciones de diferenciabilidad, aunque sean mínimas, para poder calcular el gradiente que necesitamos. Sin embargo vamos a ver que no necesitamos estrictamente que las funciones sean diferenciables, sino que extendemos al concepto de subdiferenciabilidad.

Podemos pensar en un modelo como una composición de la suma y producto de operaciones lineales con operaciones no lineales (funciones de activación), y componiendo ésta con la función de coste del modelo obtendríamos la función $f: X \times \Omega \times Y \rightarrow \mathbb{R^+}$, que recibe los pesos del modelo, los datos de entrada y sus etiquetas correctas para proporcionar el error del modelo. Esta es la función que necesitaríamos que fuera diferenciable. Las operaciones lineales preservan la diferenciabilidad, y la composición de funciones diferenciables es diferenciable por lo que si la función de pérdida y las funciones de activación son diferenciables, no tendremos ningún problema a la hora de calcular el gradiente.

Las funciones de coste son diferenciables de manera general, y la más común para problemas de clasificación es \textit{CrossEntropyLoss}, mientras que para regresión son comunes el error cuadrático medio y el error absoluto medio.

\begin{itemize}

    \item \textbf{ECM:} $\frac{1}{N} \sum_{i=1}^N \left (y_i - \hat{y_i} \right ) ^2$ 

    \item \textbf{EAM:} $\frac{1}{N} \sum_{i=1}^N \lvert y_i - \hat{y_i} \rvert$ 	

    \item \textbf{\textit{CrossEntropyLoss}:} $  - \sum_c \hat{y}_{i,c} log(\frac{e^{y_{i,c}}}{\sum_{c'=1}^C e^{y_{i,c'}}})$
\end{itemize}

En regresión $\hat{y}_i$ es el valor real  e $y_i$ es el predicho por el modelo para el dato $i$ que será un real en ambos casos. En clasificación $\hat{y}_{i,c}$ es la etiqueta real del dato $i$ para la clase $c$, que valdrá 1 en caso de que el dato pertenece a la clase $c$ y 0 en caso contrario, e $y_{i,c} \in [0,1]$ representa la probabilidad predicha por el modelo de que el dato $i$ pertenezca a la clase $c$. Finalmente $N$ es el número de datos y $C$ el número de clases. 


Hasta el año 2010, las funciones de activación más comunes para las capas ocultas eran la función sigmoide y la tangente hiperbólica. Estas funciones son diferenciables por lo que su uso no suponía ningún problema en la aplicación del descenso de gradiente. Sobre ese año se empezó a popularizar la función de activación ReLU (Rectified Linear Unit), gracias a su simplicidad, reducción de coste computacional y su aparición en modelos ganadores de competiciones de ImageNet como AlexNet en 2012. Desde entonces esta función, junto a algunas de sus variantes que aparecen en la figura \ref{fig:3.ReLU} son ampliamente usadas y con buenos resultados. Sin embargo salta a la vista que esta función no es diferenciable.


\begin{figure}
    \centering
    \includegraphics[width=0.5\linewidth]{3ReLU&oth.jpg}
    \caption{Función ReLU y algunas de sus variantes más usadas como funciones de activación.\footnote{https://www.researchgate.net/publication/319438080\_A\_novel\_softplus\_linear\_unit\_for\_deep\_convolutional\_neural\_networks}}
    \label{fig:3.ReLU}
\end{figure}



Vamos a presentar entonces el concepto de subgradiente junto con algunas de sus propiedades, obtenidas de \cite{convexSubgrad}, para ver que será una extensión del gradiente que nos permitirá usar el método de gradiente descendente con funciones que no sean diferenciables en algunos puntos pero que sí sean subdiferenciables.

\begin{definicion}[Subgradiente]
     Sea $A \subset \mathbb{R}^n$ y $f:A \rightarrow \mathbb{R}$, $g \in \mathbb{R}^n$ es un subgradiente de $f$ en $a \in A$ si existe un entorno de $a$ $U_a$ tal que $\forall y \in U_a$ se tiene:

    $$f(a)-f(y) \leq g^T(a-y)$$
    El conjunto de los subgradientes de $f$ en $a$ se denota por $\partial f(a)$. Si existe el subgradiente de $f$ en a, decimos que $f$ es subdiferenciable en $a$.
\end{definicion}

Necesitamos también un comportamiento similar al de las funciones diferenciables, en particular necesitamos que las funciones subdiferenciables se preserven a través de las operaciones de suma, multiplicación por escalares y composición.

\begin{enumerate}

	\item{\textbf{Multiplicación escalar no negativa}: $\partial (af) = a \cdot \partial f , a\geq0$
	
	Por definición $g$ es un subgradiente de $f$ en $x_0$ si:
	
	$$f(x) \geq f(x_0) + g^T(x-x_0), \quad \forall x \in dom(f).$$

	Multiplicando la desigualdad por $c \geq 0$:

	$$cf(x) \geq cf(x_0) + cg^T(x-x_0), \quad \forall x \in dom(f).$$

	Por tanto $cg$ es un subgradiente de $h(x)=cf(x)$ en $x_0$.

	}
	
	\item{ \textbf{Suma}: $\partial (f_1+f_2)(x) = \partial f_1(x) + \partial f_2(x)$

	Sea $g_1$ un subgradiente de $f_1$ y $g_2$ un subgradiente de $f_2$, considerando el punto $x_0 \in dom(f_1) \cap dom(f_2)$, por definición tenemos:

	$$f_1(x) \geq f_1(x_0) + g_1^T(x-x_0), \quad \forall x \in U_{x_0},$$

	$$f_2(x) \geq g_2(x_0) + g_2^T(x-x_0), \quad \forall x \in U_{x_0}.$$

	Sumamos las dos desigualdades para obtener que $g_1 + g_2$ es un subgradiente de $(f_1 + f_2)(x_0)$:

	$$f_1(x) + f_2(x) \geq f_1(x_0) + f_2(x_0) + \left ( g_1 + g_2 \right ) ^T \left ( x - x_0 \right ).$$


	}
	
	\item{ \textbf{Composición afín}: Si $h(x)=f(Ax + b) \Rightarrow \partial h(x)= A^T \partial f(Ax+b)$.
	
	Tenemos que $g$ es un subgradiente de $f$ en $y_0$:
	
	$$f(y) \geq f(y_0) + g^T(y-y_0), \quad \forall y \in U_{y_0}.$$

	Tomamos $y=Ax + b$ y por tanto $y_0= Ax_0 + b$. Sustituyendo:

	$$f(Ax + b) \geq f(Ax_0 + b) + g^T(Ax + b - (Ax_0 + b)),$$

	$$h(x)=f(Ax+b) \geq h(x_0) + g^TA(x-x_0).$$

	Por tanto $A^Tg$ es un subgradiente de $h(x)=f(Ax+b)$ en $x_0$.

	}
	
\end{enumerate}


Tenemos que comprobar que el subgradiente extiende al gradiente, es decir, que cuando existe gradiente entonces existe un único subgradiente y coincide con él. Además hay funciones que no son diferenciables pero sí subdiferenciables. Esto último se hace evidente con el ejemplo \ref{ej:RELUsub} de la función ReLU. Vamos a demostrar por tanto que si $f: X \subseteq \mathbb{R}^n \rightarrow \mathbb{R}^m$ es diferenciable en el punto $x\in X$ entonces  $\partial f(x)= \left \{ \nabla f(x) \right \}$.

Como $f$ es diferenciable en $x$, existe un entorno $U_x$ de $x$ donde el gradiente satiface 

$$  f(y) = f(x) + \nabla f(x)^T(y-x) + o(\|y-x \|) \quad \forall y \in U_x .$$

Por tanto tenemos

$$ f(y) \geq f(x) + \nabla f(x)^T(y-x) \quad \forall y \in U_x.$$

Es decir que el gradiente de $f$ en $x$ es también un subgradiente de $f$ en $x$. Tenemos que $\nabla f(x) \in \partial f(x)$, nos queda demostrar que es único. Para ello vamos a suponer que existe otro subgradiente de $f$ en $x$, $g \in \partial f(x)$. Sea $u=x+tw$, definimos la función

$$\phi(t)=f(u)-f(x)- g^T(u-x) = f(x+tw) -f(x) - g^T(tw) \geq 0$$

donde se ha usado que $g \in \partial f(x)$ para ver que es no negativa. Derivamos la función para obtener $\phi'(t) = \nabla f(x)^Tw - g^Tw$. Vemos que para $t=0$ se tiene que $\phi(0)=0$, con lo que hay un mínimo en ese punto. Tenemos por tanto que $\phi'(0)=0$ o equivalentemente $\nabla f(x)^Tw=g^Tw$. Como $w$ es arbitrario, concluimos que $g=\nabla f(x)$. Como el gradiente es un subgradiente, y todo subgradiente coincide con él, se tiene que es el único subgradiente de $f$ en $x$, $\partial f(x) = \left \{ \nabla f(x) \right \}$. 



Para presentar una proposición que relaciona los subgradientes con las funciones convexas, las cuales están muy ligadas a la convergencia del gradiente descendente, primero vamos a definir lo que es un conjunto convexo.


\begin{definicion}[Conjunto convexo]
    Un subconjunto $E \subseteq \mathbb{R}^n$ es convexo cuando, para cualesquiera dos puntos de $E$, el segmento que los une está contenido en $E$:

    $$x,y \in E \Rightarrow \left \{ (1-t)x + ty : t\in [0,1] \right \} \subset E.$$
\end{definicion}




\begin{definicion}[Función convexa]
    Sea $E \subset \mathbb{R}^n$ un conjunto convexo no vacío y sea $f:E \rightarrow \mathbb{R}$, $f$ es una función convexa en $E$ si, y solo si:

    $$f( (1-t)y + tx) \leq (1-t) f(y) + tf(x), \quad \forall t \in [0,1], \forall x,y \in E.$$
\end{definicion}

\begin{proposicion}[Existencia de subgradientes]
\label{prop:subgrad}
    Sea $E \subset \mathbb{R}^n$ un conjunto convexo y $f:E \rightarrow \mathbb{R}$. Si $\forall x \in E, \partial f(x) \neq \emptyset$ entonces $f$ es una función convexa. Recíprocamente, si $f$ es convexa  entonces se tiene que $\forall x \in int(E)$ $\partial f(x) \neq \emptyset$.
\end{proposicion}

Esta proposición nos asegura que la familia de las funciones ReLU, que son convexas, siempre tienen subgradiente en su interior. Para demostrarla, primero vamos a necesitar de un teorema, en el ámbito de la convexidad:

\begin{teorema}[Teorema del Hiperplano de apoyo]
    Sea $E \subset \mathbb{R}^n$ un conjunto convexo y $x_0 \in Fr(E)$ un puntro de la frontera de $E$. Entonces, $\exists w \in \mathbb{R}^n, w \neq 0$ tal que
    $$\forall x \in E, \quad w^Tx \geq w^T x_0$$
\end{teorema}

\textbf{\textit{Demostración de la proposición 3.1.}}
Para la primera implicación, queremos probar que si para todo punto $x \in E$ existe al menos un subgradiente de $f(x)$ entonces se verifica
$$f((1-t)x+ty) \leq (1-x)f(x)+tf(y) \quad \forall x,y \in E, t \in [0,1],$$
es decir, que $f$ es convexa. Tomamos $g \in \partial(z),$ $z \in E$ ya que $\forall z \in E,  \partial f(z) \neq \emptyset $ , y tenemos por la definición de subgradiente:

\begin{align}
	f(z) - f(x) \leq g^T(z-x), \notag \\
	f(z) - f(y) \leq g^T(z-y), \label{eq:des2}
\end{align}

para $x,y \in E$. Tomamos $z=(1-t)x + ty$ y sustituimos:

\begin{align}	
	f((1-t)x + ty) - f(x) &\leq g^T(((1-t)x + ty)-x), \notag \\
	f((1-t)x + ty) + g^T(x - ((1-t)x-ty)) &\leq f(x), \notag \\
	f((1-t)x + ty) + g^T(t(x-y)) &\leq f(x), \notag \\
	f((1-t)x + ty) + tg^T(x-y) &\leq f(x). \label{proof1:f(x)}
\end{align}


Desarrollando en \ref{eq:des2} de manera análoga obtenemos

\begin{equation}\label{proof1:f(y)}
    f((1-t)x + ty) + (1-t)g^T(y-x)) \leq f(y).
\end{equation}

Ahora multiplicamos la desigualdad \ref{proof1:f(x)} por $(1-t)$ y la \ref{proof1:f(y)} por $t$, y de su suma obtenemos:

\begin{align*}
	(1-t)f(x) + tf(y) &\geq  (1-t)f((1-t)x+ty) + t(1-t)g^T(x-y) \\
	&+ tf((1-t)x+ty) + t(1-t)g^T(y-x) \\
	 &= f((1-t)x + ty) + t(1-t) g^T(x-y) + t(1-t)g^T(y-x) \\
	&= f((1-t)x+ty)
\end{align*}

donde se ha usado que $g^T(x-y) + g^T(y-x)=0$. Entonces tenemos que $(1-t)f(x) + tf(y) \geq f((1-t)x + ty), $ $ \forall x,y \in E, $ $ t \in [0,1]$. Por tanto $f$ es convexa, como queríamos probar.

Ahora vamos a probar que $f$ tiene algún subgradiente en $int(E)$ si es convexa. Definimos el epigrafo de una función $f$ como

 $$epi(f)=\left \{ (x,t) \in E \times \mathbb{R} : t \geq f(x) \right \}.$$

Es obvio que f es convexa si y sólo si su epigrafo es un conjunto convexo. Vamos a aprovechar esta propidad y vamos a construir un subgradiente usando un hiperplano de apoyo al epigrafo de la función. Sea $x \in E$, claramente $(x, f(x)) \in Fr(epi(f))$, y $epi(f)$ es un conjunto convexo por ser $f$ convexa. Entonces usando el Teorema del Hiperplano de Apoyo, existe $(a,b) \in \mathbb{R}^n \times \mathbb{R}$ tal que

\begin{equation}\label{proof1:epi}
    a^Tx + bf(x) \geq a^Ty + bt, \quad \forall (y,t) \in epi(f).
\end{equation}

Reordenando tenemos

$$b(f(x)-t) \geq a^Ty - a^Tx.$$

Como $t \in [f(x), + \infty [$, para que se mantenga la igualdad incluso cuando $t \rightarrow \infty$, debe ocurrir que $b\leq 0$. Ahora vamos a asumir que $x \in int(E)$. Entonces tomamos $\epsilon > 0$, verificando que $y=x+\epsilon a \in E$, lo que implica que $b\neq 0$, ya que si $b=0$ entonces necesariamente $a=0$. Reescribiendo \ref{proof1:epi} con $t=f(y)$ obtenemos

$$f(x) - f(y) \leq \frac{1}{|b|} a^T (x-y).$$

Por tanto $\frac{a}{|b|} \in \partial f(x)$, lo que demuestra la otra parte de la implicación.

\begin{comment}
	
	Para la última parte, sea $f$ una función convexa y diferenciable. Entonces por definición de convexidad para $t \in [0,1]$ y $x,y \in E$ tenemos
	
	$$tf(y) + (1-t)f(x) \geq f((1-t)x + ty).$$
	
	Deducimos entonces que
	
	\begin{align*}
		f(y) &\geq \frac{f((1-t)x + ty) - (1-t)f(x)}{t}
		
		&= f(x) + \frac{f(x + t(y-x)) - f(x)}{t}
	
		&\xrightarrow[t \rightarrow 0]{} \quad f(x) + \nabla f(x)^T (y-x).
	\end{align*}
	
	
	
	Lo que demuestra que $\nabla f(x) \in \partial f(x)$.
\end{comment}


\begin{flushright}
    $\square$
\end{flushright} 





Tenemos entonces que el subgradiente es una extensión del gradiente en aquellos puntos que no son diferenciables. Por ello podríamos decir que existe el método de descenso de subgradiente, que permite usar funciones que no son diferenciables en todos los puntos, y que se usa de manera implícita en el momento en el que en un modelo se usan funciones de la familia ReLU. Conviene destacar esta diferencia para no perder la rigurosidad, aunque solo sea una formalidad, ya que realmente no se hacen diferencias entre uno y otro método, así que nos seguiremos refiriendo al método de descenso de gradiente aunque estemos trabajando con subgradientes. En la práctica simplemente se elige un valor predeterminado para la derivada en el punto que estas funciones no son diferenciables.

\begin{ejemplo}[Subgradiente de la función ReLU]\label{ej:RELUsub}
     La función ReLU es continua en todo el dominio y diferenciable en $]-\infty,0[ \cup ]0,\infty[$. Su subgradiente es el siguiente:

    $$ \nabla ReLU(x)=\left\{\begin{matrix}
1, \quad si \quad x \in ]0,\infty[ \\
c \in [0,1] \quad si \quad x=0\\
0 \quad si \quad x \in ]-\infty,0[
\end{matrix}\right.$$
\end{ejemplo}

En \cite{ReLuat0} se analiza la elección del valor que toma el subgradiente en el punto $x=0$ y se ve su influencia, que no es poca, en la ejecución del algoritmo, y se concluye que el valor 0 es el que ofrece mejor robustez de manera general.




\subsection{Convergencia} \label{sec:convergencia}

La convergencia es un factor crucial en el algoritmo de gradiente descendente. Al tratarse de un algoritmo de optimización iterativo, iremos buscando el mínimo global de la función de coste en varios pasos, o en su defecto un mínimo local que nos ofrezca una solución subóptima. El algoritmo se mueve hacia puntos de menor gradiente por lo que en caso de converger lo hará a puntos donde sea 0. Un factor clave para la convergencia será el hecho de que la función de pérdida sea o no una función convexa.

En caso de que lo sea sólo existirá un punto crítico\footnote{Si existiera una región donde la función fuera constante, cada punto de la región sería un punto crítico pero esto sería extraordinariamente extraño} y será un mínimo global, por lo que no tenemos que preocuparnos de si el algoritmo se queda estancado en un mínimo local, ya que si converge tendremos la solución óptima. Además en este caso el análisis de la convergencia resulta mucho más sencillo, y por eso encontramos más resultados teóricos y más fuertes que en el caso contrario. Desgraciadamente la situación normal es que la función de coste no sea convexa, y de hecho comprobar que una función sea convexa se trata de un problema NP-Hard \cite{Ahmadi_2011_NP_Convex}, por lo que en la práctia normalmente no realizamos el análisis teórico de la función y la convergencia previo al entrenamiento del modelo. En caso que no sea convexa, podemos converger hacia un punto crítico que no sea un mínimo global, con lo cual el algoritmo parará y puede que hallamos llegado a una solución que aunque sea subóptima no sea lo suficientemente buena.


\subsubsection{Convergencia para BGD}

Los desarrollos teóricos sobre la convergencia del algoritmo de descenso del gradiente son muchos y variados, sin embargo no son lo suficientemente útiles en la práctica y se presupone que la función no es convexa. Los principales inconvenientes para el desarrollo de un marco teórico práctico son:

\begin{itemize}

    \item No existen resultados generales que nos permitan conocer el comportamiento de la convergencia del algoritmo en el problema que estemos tratando con un coste asequible. Los resultados son muy específicos y dependen de la función de coste, el valor de los hiperparámetros y la versión del algoritmo de gradiente descendente que estemos utilizando.

    \item El estudio teórico de la función de coste es muy complejo y requiere muchos recursos computacionales. Por lo tanto la tendencia a nivel experimental es invertir esos recursos en el entrenamiento, ya que ofrece mejores resultados en relación coste/beneficio de manera general que el estudio teórico de los elementos del algoritmo. Además es un procedimiento genérico aplicable en cualquier problema, por lo que resulta más sencillo.

   
\end{itemize}


En el caso que la función de coste sea convexa tenemos un caso más sencillo de analizar, principalmente debido a la curvatura que tienen las funciones convexas y al hecho de que cualquier punto crítico será un mínimo global.





\begin{teorema}[Convergencia para funciones convexas]\label{proof:gdconvex}
    Suponemos la función $f: \mathbb{R}^n \rightarrow \mathbb{R}$ convexa y diferenciable, con su gradiente Lipschitz continuo con constante $L>0$, $\| \nabla f(x) - \nabla f(y) \|_2 \leq L \|x-y\|_2 \quad \forall x, y \in \mathbb{R}^n$. Si ejecutamos el algoritmo de gradiente descendente $k$ iteraciones con un $\eta<1/L$ constante, el error disminuirá tras cada iteración, llegando a una solución $x^{(k)}$ que satisface la siguiente desigualdad:

    $$f(x^{(k)})-f(x^*) \leq \frac{\|x^{(0)}-x^* \|^2_2}{2\eta k}$$

    donde $x^*$ es el mínimo global de la función de error. 
\end{teorema}

\vspace{1cm}

\begin{flushleft}
   \textbf{\textit{Demostración.}}
\end{flushleft} 



En el teorema anterior $x \in \mathbb{R}^n$ contiene los pesos del modelo, y suponemos que el conjunto de datos con el que entrenamos es constante, por lo tanto el error del modelo, $f(x)$, sólo dependerá de los parámetros $x$.

Como el gradiente $\nabla f$ es Lipschitz continuo con constante $L$ entonces $\nabla ^2 f(x) \preceq LI$. Esto equivale a que $ \textstyle LI- \nabla^2f(x) $ sea una matriz semidefinida positiva, por lo que $\nabla ^2 f(x) -LI$ es una matriz semidefinida negativa. Ahora hacemos un desarrollo cuadrático de $f$ alrededor de $f(x)$ para obtener:

\begin{align*}
    f(y) &\leq f(x) + \nabla f(x)^T (y-x) +\frac{1}{2}\nabla^2 f(x) \|y-x\|^2_2  \\
    &\leq f(x) + \nabla f(x)^T(y-x) + \frac{1}{2}L \|y - x \|^2_2.
\end{align*}

Consideramos ahora $y$ como la actualización de los pesos del gradiente descendente, $y=x - \eta \nabla f(x)=x^+$. 


\begin{align*}
    f(x^+) &\leq f(x) + \nabla f(x)^T(x^+-x) + \frac{1}{2}L \|x^+ - x \|^2_2 \\
    &= f(x) + \nabla f(x)^T(x - \eta \nabla f(x) -x) + \frac{1}{2}L \|x - \eta \nabla f(x) - x \|^2_2 \\
    &= f(x) - \eta \nabla f(x)^T \nabla f(x) + \frac{1}{2} L \| \eta \nabla f(x) \|^2_2 \\
    &= f(x) - \eta \| \nabla f(x) \|^2_2 + \frac{1}{2} L \eta^2 \| \nabla f(x) \|^2_2 \\
    &= f(x) - (1- \frac{1}{2}L \eta) \eta \| \nabla f(x) \|^2_2.
\end{align*}

Usamos $\eta \leq \frac{1}{L}$ para ver que $-(1-\frac{1}{2}L \eta)= \frac{1}{2} L \eta - 1 \leq \frac{1}{2} L (\frac{1}{L}) - 1 = -\frac{1}{2}$, y sustituyendo esta expresión en la desigualdad anterior obtenemos 

\begin{equation}\label{eq:gdproof1}
	f(x^+) \leq f(x) - \frac{1}{2} \eta \| \nabla f(x) \|^2_2 .
\end{equation}

Esta última desigualdad se traduce en que tras cada iteración del algoritmo del descenso de gradiente el valor del error del modelo es estrictamente decreciente, ya que el valor de $\frac{1}{2} \eta \| \nabla f(x) \|^2_2$ siempre es mayor que 0 a no ser que $\nabla f(x)=0$, en cuyo caso habremos encontrado el óptimo. 

Ahora vamos a acotar el valor del error en la siguiente iteración, $f(x^+)$, en términos del valor óptimo de error $f(x^*)$. Como $f$ es una función convexa se tiene

\begin{align*}
    f(x) &\leq f(x^*) + \nabla f(x)^T (x-x^*).
\end{align*}

Sustituyendo en \ref{eq:gdproof1} obtenemos

\begin{align*}
    f(x^+) &\leq f(x^*) + \nabla f(x)^T (x-x^*) - \frac{\eta}{2} \| \nabla f(x) \| ^2_2 \\ 
    f(x^+) - f(x^*) &\leq  \frac{1}{2\eta}  \left ( 2 \eta \nabla f(x)^T (x-x^*) - \eta ^2 \| \nabla f(x) \| ^2_2 \right ) \\ 
    f(x^+) - f(x^*) &\leq  \frac{1}{2\eta}  \left ( 2 \eta \nabla f(x)^T (x-x^*) - \eta ^2 \| \nabla f(x) \| ^2_2 - \| x - x^* \|^2_2 + \| x - x^* \|^2_2 \right ).    
\end{align*}

Como $  2 \eta \nabla f(x)^T (x-x^*) - \eta ^2 \| \nabla f(x) \| ^2_2 - \| x - x^* \|^2_2 = \| x - \eta \nabla f(x) - x^* \|^2_2 $, se tiene que

$$ f(x^+) - f(x^*) \leq  \frac{1}{2\eta}  \left ( \| x - x^* \|^2_2 -  \| x - \eta \nabla f(x) - x^* \|^2_2 \right ) .$$

Usamos ahora la definición de $x^+$ en esta última desigualdad

$$ f(x^+) - f(x^*) \leq  \frac{1}{2\eta}  \left ( \| x - x^* \|^2_2 -  \| x^+ - x^* \|^2_2 \right ) .$$

Hacemos la sumatoria sobre las $k$ primeras iteraciones y tenemos

\begin{align*}
    \sum^k_{i=1} \left ( fx^{(i)} - f(x^*) \right ) &\leq \sum^k_{i=1} \frac{1}{2\eta}  \left ( \| x^{(i-1)} - x^* \|^2_2 -  \| x^{(i)} - x^* \|^2_2 \right ) \\ 
    &=\frac{1}{2\eta}  \left ( \| x^{(0)} - x^* \|^2_2 -  \| x^{(k)} - x^* \|^2_2 \right ) \\ 
    &\leq \frac{1}{2\eta}  \left ( \| x^{(0)} - x^* \|^2_2 \right ). 
\end{align*}

El sumatorio de la derecha ha desaparecido ya que es una serie telescópica. Usando que $f$ decrece con cada iteración, e introduciendo la anterior desigualdad, finalmente llegamos a donde queríamos:

$$f(x^{(k)}) - f(x^*) \leq \frac{1}{k} \sum ^k _{i=1} \left ( f(x^{(i)} - f(x^*) \right ) \leq \frac{\|x^{(0)}-x^* \|^2_2}{2\eta k} .$$


\begin{flushright}
    $\square$
\end{flushright} 


Este teorema nos garantiza que bajo las condiciones supuestas el algoritmo del gradiente descendente converge y además lo hace con ratio de convergencia de $O(1/k)$. Es un resultado teórico muy fuerte que por desgracia no puede usarse en la práctica en la gran mayoría de casos: la constante de Lipschitz $L$ es computacionalmente costosa de calcular, por lo que se usan aproximaciones experimentales para el $\eta$, además en muy contadas ocasiones la función de error con la que trabajamos es convexa, y tampoco es sencilla de calcular por lo que directamente no se comprueba si lo es o no lo es, y directamente la suponemos no convexa. 



\subsubsection{Convergencia para SGD y MBGD}


Podemos obtener un resultado mucho más práctico, ya que es para SGD y MBGD y además con condiciones más relajadas. A partir de ahora usaremos SGD para referirnos de manera general a las versiones estocásticas del algoritmo de gradiente descendente, tanto SGD como MBGD.Usando la teoría de algoritmos aproximados estocásticos, con el teorema de Robbins-Siegmund tenemos que bajo las siguientes condiciones, cuando la función es convexa se tiene la convergencia casi segura al mínimo global y cuando no lo es hay convergencia casi segura a un mínimo local. Esto nos da un criterio sencillo con el que aseguramos la convergencia, y que no depende de parámetros como la constante de Lipschitz que son complejos de computar. Para una correcta exposición del teorema y su demostración, primero vamos a definir el concepto de supermartingala y casi-supermartingala. 

Usamos como referencia el libro \url{http://www.sze.hu/~harmati/Sztochasztikus\%20folyamatok/lawler.pdf}. En primer lugar vamos a introducir los conceptos de martingala, supermartingala y casi supermartingala, que son tipos de procesos estocásticos. Luego enunciaremos un teorema, el de Robbins-Siegmund, que proporciona un fuerte resultado de convergencia para los procesos casi supermartingalas. Demostrando que el algoritmo de SGD es un proceso de este tipo precisamente, enunciaremos el teorema de convergencia para estos algoritmos, demostrándolo en gran parte gracias al teorema de Robbins-Siegmund.





Vamos a hacer la notación un poco más compacta. Si $X_1, X_2, \ldots$ es una sucesión de variables aleatorias usaremos $\mathcal{F}_n$ para denotar "la información contenida en $X_1, \ldots, X_n$". Escribiremos $E[Y | \mathcal{F}_n]$ en lugar de $E[Y | X_1, \ldots, X_n]$.

Una martingala es un modelo de juego justo. Denotamos por $\left \{ \mathcal{F}_n \right \}$ una sucesión creciente de información, es decir, para cada $n$ tenemos una sucesión de variables aleatorias $\mathcal{A}_n$ tal que $\mathcal{A}_m < \mathcal{A}_n$ si $m<n$. La información que tenemos en el momento $n$ es el valor de todas las variables en $\mathcal{A}_n$. La suposición $\mathcal{A}_m \subseteq \mathcal{A}_n$ implica que no perdemos información. Decimos que una variable aleatoria $X$ es $\mathcal{F}_n$-medible si podemos determinar el valor de $X$ en caso de conocer el valor de todas las variables aleatorias en $\mathcal{A}_n$. A menudo está sucesión creciente de información $\mathcal{F}_n$ se denomina filtración.

Decimos que una secuencia de variables aleatorias $M_0, M_1, M_2, \ldots$ con $E[ |M_i|] < \infty$ es una martingala con respecto a $\left \{ \mathcal{F} \right \} _n$ si cada $M_n$ es medible con respecto a $\mathcal{F}_n$ y para cada $m<n$,

\begin{equation}\label{eq:martingala}
	E[M_n | \mathcal{F}_m] = M_n,
\end{equation}

o equivalentemente,

\begin{equation}
	E[M_n - M_m |\mathcal{F}_m] = 0.
\end{equation}	

La condición $E[|M_i|] < \infty$ es necesaria para garantizar que las esperanzas condicionadas están bien definidas. Si $\mathcal{F}_n$ es la información en variables aleatorias $X_1, \ldots, X_n$ entonces también diremos que $M_0, M_1, \ldots$ es una martingala con respecto a $X_0, X_1, \ldots$. A veces diremos que $M_0, M_1, \ldots$ es una martingala sin hacer referencia a la filtración $\mathcal{F}_n$. En ese caso significará que la sucesión $M_n$ es una martingala con respecto a sí misma. 

Un proceso $M_n$ con $E[|M_n|] < \infty$ es una supermartingala con respecto a $\left \{ \mathcal{F}_n \right \}$ si para cada $m<n$ se tiene $E[M_n | \mathcal{F}_m) \leq M_m$. En otras palabras, una supermartingala es un juego injusto. Si un proceso no negativo no verifica la desigualdad anterior, pero verifica que 

\begin{equation*}
	E[M_n | \mathcal{F}_m] \leq (1 + \beta _m ) V_m + \xi _m + \zeta _m
\end{equation*}

para $m<n$ y $\beta _n, \xi _n, \zeta _n \geq 0$ siendo $\mathcal{F}_n$-medibles, decimos entonces que $M_n$ es una casi supermartingala. Usando el teorema de Robbins-Siegmund, vamos a obtener un poderoso resultado de convergencia para procesos estocásticos no negativos que son casi supermartingalas.

\begin{teorema}[Teorema de Robbins-Siegmund]\label{teor:sig}
	Suponemos que $V_n$ es una casi supermartingala no negativa. Si 

	$$ \sum_{n=1}^{\infty} \beta_n < \infty \quad \text{y} \quad \sum_{n=1}^{\infty} \xi_n < \infty \quad \text{casi seguro,}$$

	entonces existe una variable aleatoria no negativa $V_{\infty}$ que verifica

	$$ \displaystyle \lim_{n \to \infty}V_n = V_{\infty} \quad \text{y} \quad \sum_{n=1}^{\infty} \zeta < \infty \quad \text{casi seguro.}$$
\end{teorema}


Ahora vamos a comprobar que el algoritmo de gradiente descendente en su versión SGD es una casi supermartingala. Vamos a definir las funciones fuertemente convexas, porque las necesitaremos para este desarrollo. 

\begin{definicion}[Función estrictamente convexa]
	Sea $E \subset \mathbb{R}^n$ un conjunto convexo no vacío y sea $f:E \rightarrow \mathbb{R}$, $f$ es una función fuertemente convexa en $E$ si es estrictamente convexa\footnote{La convexidad estricta es igual que la convexidad normal, pero la desigualdad de su definición es una desigualdad estricta} y además se verifica para algún $m \geq 0$:

   	$$( \nabla f(x) - \nabla f(y))^T(x-y) \geq m \|x-y\|^2 \quad  \forall x,y \in E.$$
\end{definicion}

Ahora nos fijamos en la minimización, para un conjunto abierto de parámetros $\mathcal W$, de la función objetivo

\begin{equation}
	H(W) = E[ C(X,W)]
\end{equation}

para una función de pérdida C. Asumimos que $\nabla H(W) = E[\nabla C(X, W)]$ y que 

\begin{equation}
	G(W):= E[ \| \nabla C(X,W)\|^2] \leq A + B\|W\|^2.
\end{equation}

Asumimos también que $\nabla H(W^*)=0$ y que

\begin{equation}
	(W-W^*)^T \nabla H(W) \geq c \|W - W^*\|^2,
\end{equation}

lo que implica que $W^*$ es un minimizador único de $H$, es decir, que es un mínimo global y es único. Esta última condición se mantiene siempre que $H$ sea estrictamente convexa, pero solo necesitamos que se mantenga en $W^*$.

Añadiendo la tasa de aprendizaje como una sucesión en lugar de una constante y expresando las sucesiones como en \ref{teor:sig} para facilitar la comprensión, la regla de actualización de los pesos que vimos en \ref{eq:SGD} quedaría como

\begin{equation}
W_{n} = W_{n-1} - \eta_{n-1} \nabla C(X_n, W_{n-1})
\end{equation} 

para una secuencia $X_1, X_2, \ldots$ de variables aleatorias independientes e idénticamente distribuidas. Asumimos que la tasa de aprendizaje $\eta_{n-1}$ puede depender de $X_1, \ldots, X_{n-1}$ y $W_0, \ldots, W_{n-1}$. De esto obtenemos que 

\begin{align*}
	V_{n} &= \| W_{n} - W* \|^2 \\
	       &= \| W_{n-1} - \eta_{n-1} \nabla C(X_{n}, W_{n-1}) - W^* \|^2 \\
	       &= V_{n-1} + \eta^2_{n-1} \|\nabla C(X_n, W_{n-1}) \|^2 - 2\eta_{n-1}(W_{n-1} - W^*)^T \nabla C(X_n, W_{n-1}).
\end{align*}

Tomando la esperanza condicionada resulta

\begin{align}
	E[V_n | \mathcal{F}_{n-1}] &= V_{n-1} + \eta^2_{n-1} E[ \| \nabla C(X_n, W_{n-1}) \|^2 | \mathcal{F}_{n-1}] \notag \\
	& \quad - 2\eta_{n-1}(W_{n-1} - W^*)^T E[ \nabla C(X_n, W_{n-1}) | \mathcal{F}_{n-1}] \notag \\  
	&= V_{n-1} + \eta^2_{n-1} G(W_{n-1}) - 2 \eta_{n-1}(W_{n-1} - W^*)^T \nabla H(W_{n-1}) \notag \\
	&\leq V_{n-1} + \eta^2_{n-1} (A+B \|W_{n-1} \|^2) - 2c \eta_{n-1} \|W_{n-1} - W^*\|^2. \label{eq:sig1}
\end{align}

Ahora observamos que 

\begin{align*}
	\| W_{n-1} \|^2 &= \| W_{n-1} - W^* + W^* \|^2 \\
			     &\leq (\| W_{n-1} - W^* \| + \|W^* \|)^2 \\
			     & \leq 2 \| W_{n-1} - W^* \|^2 + 2 \|W^* \|^2 \\
			     &= 2V_{n-1} + 2 \|W^* \|^2,
\end{align*}

e introduciendo esta desigualdad en \ref{eq:sig1} obtenemos

\begin{align*}
	E[V_n | \mathcal{F}_{n-1} ] &\leq V_{n-1} + \eta^2_{n-1} (A+ 2BV_{n-1} + 2B \|W^* \|^2) - 2c\eta_{n-1} V_{n-1} \\
					&= (1 + 2B \eta_{n-1}^2) V_{n-1} + \eta^2_{n-1} (A + 2B \| W^* \|^2) - 2c \eta_{n-1} V_{n-1}.
\end{align*}

Esto demuestra que $V_n$ es una casi supermartingala con $\beta_n=2B\eta^2_n$, $\xi_n = \eta_n^2(A + 2B \| W^* \|^2)$ y $\zeta_n = c \eta_n V_n$.

Estamos ya en condiciones de enunciar y demostrar el teorema relativo a la convergencia del algoritmo SGD.

\begin{teorema}[Convergencia de algoritmos SGD]\label{teor:convsgd}
	Con las suposiciones realizadas anteriormente sobre la función de coste $C$, el proceso $V_n$ converge casi seguro a un límite $V_{\infty}$ si 

	\begin{equation}\label{eq:convsgd1}
		\sum_{n=1}^{\infty} \eta^2_{n} < \infty \quad \text{casi seguro.}
	\end{equation}

	Si también
	\begin{equation}\label{eq:convsgd2}
		\sum_{n=1}^{\infty} \eta_{n} = \infty \quad \text{casi seguro}
	\end{equation}

	entonces el límite es $V_{\infty}=0$ y se tiene

	\begin{equation*}
		\displaystyle \lim_{n \to \infty} W_n = W^* \quad \text{casi seguro.}
	\end{equation*}
\end{teorema}


\begin{flushleft}
   \textbf{\textit{Demostración.}}
\end{flushleft} 

La primera parte se sigue directamente del teorema de Robbins-Siegmund. Para la segunda, asumimos $V_{\infty} > 0 $ en un conjunto de probabilidad positiva y procedemos por contradicción. Hay entonces una variable aleatoria $N$ tal que en ese conjunto $V_n \geq \frac{V_{\infty}}{2}$ para $n \geq N$, y 

\begin{equation*}
	\sum_{n=1}^{\infty} \zeta _n = c \sum_{n=1}^{\infty} \eta _n V_n \geq \frac{cV_{\infty}}{2} \sum_{n=N}^{\infty} \eta _n = \infty
\end{equation*}

con probabilidad positiva. Esto contradice el teorema de Robbins-Siegmund. Concluimos que $V_{\infty}=0$ casi seguro, entonces $V_n= \| W_n - W^* \| ^2 \rightarrow 0$ casi seguro, o 

\begin{equation*}
	W_n \rightarrow W^*
\end{equation*}

casi seguro cuando $n \rightarrow \infty$.
\begin{flushright}
    $\square$
\end{flushright} 

Aunque la suposición de que $H$ es globalmente fuertemente convexa es demasiado fuerte, ya que hace que el teorema no sea útil en la práctica, podemos esperar que el algoritmo tenga un comportamiento similar en el entorno de un minimizador local $W*$ si $H$ es fuertemente convexa en ese entorno.

La conclusión más importante que obtenemos de este resultado es que la tasa de aprendizaje debe tender a cero para asegurarnos la convergencia teórica del algoritmo. En el teorema \ref{teor:convsgd} la condición \ref{eq:convsgd1} nos dice cómo de rápido debe converger, mientras que la condición \ref{eq:convsgd2} nos dice que no debe converger demasiado rápido.

\begin{ejemplo}
	Tomamos como valores de la tasa de aprendizaje la sucesión $\eta _n = e^{-n}$. Entonces tenemos que el proceso $V_n$, es decir el algoritmo SGD, converge a $V_{\infty}$ ya que se cumple la primera condición del teorema. Sin embargo la segunda condición no se cumple, lo que quiere decir que no sabemos si $V_{\infty}=0$, por tanto no nos aseguramos converger a un minimizador. 
\end{ejemplo}

\subsubsection{Problemas en la convergencia}


En el teorema \ref{proof:gdconvex} tenemos asegurada la convergencia a un mínimo aunque con unos requisitos que no se suelen encontrar en la práctica. En la proposición \ref{prop:convsgd} por el contrario solo nos garantizamos llegar a un punto crítico, ni siquiera a un mínimo local. Encontramos aquí el mayor problema para la convergencia del algoritmo del gradiente descendente: la convergencia prematura en puntos con gradiente muy cercano a cero que no son soluciones subóptimas. 

Cuando el algoritmo se aproxima a un punto crítico, la magnitud del gradiente se aproxima a cero, y teniendo en cuenta la regla de actualización de los pesos, $W_{t+1}=W_t - \eta \nabla C(W)$, tenemos por tanto que $W_{t+1} - W_t \approx 0$. Es decir que las modificaciones de los pesos con las actualizaciones serán prácticamente nulas, haciendo que el algoritmo se pare o que progrese de manera muy lenta cerca de estos puntos, lo que en un primer momento podría aparentar una falsa convergencia en regiones planas por ejemplo. 

Los puntos críticos más comunes son los puntos de silla, que definimos como un punto $x_s$ que verifica que $\nabla f(x_s)=0$ pero $x_s$ no es ni un mínimo local ni un máximo local. En $x_s$ la matriz Hessiana de $f$, $\nabla^2 f(x_s)$, tiene valores propios tanto positivos como negativos, lo que indica que la función $f$ se curva hacia abajo en unas direcciones y hacia arriba en otras en el punto $x_s$.

En espacios de alta dimensionalidad, que son comunes en las redes neuronales, la probabilidad de encontrar puntos de silla es mucho mayor que la de encontrar máximos y minímos locales. Para una función $f:\mathbb{R}^n \rightarrow \mathbb{R}$, el número de puntos de silla normalmente crece exponencialmente con respecto a la dimensión $n$. Esto se debe a que la probabilidad de encontrar valores propios de ambos signos en la matriz Hessiana aumenta con la dimensionalidad del espacio de parámetros \cite{dauphin2014SaddlePoints}. 

La manera de solventar estos problemas es utilizar modificaciones en el algoritmo de gradiente descendente que proporcionan mejores propiedades a su convergencia, ya que las estrategias de SGD y MBGD ofrecen una pequeña pero insuficiente solución a este problema. Al calcular el gradiente mediante una aproximación con un subconjunto de los datos, se introduce un ruido $\epsilon$ en su cálculo con lo que $W_{t+1} - W_t \approx \epsilon$, que puede servir para conseguir escapar de ese punto de silla. Dichas modificaciones se denominan optimizadores y a diferencia de las versiones vistas en la sección \ref{sec:estrategias}, que variaban solo en la cantidad de datos usados para calcular el gradiente, estos optimizadores cambian la regla de actualización de los pesos añadiendo nuevos cálculos, hiperparámetros y estrategias para conseguir que el algoritmo mejore en estabilidad, robustez y velocidad de convergencia.


Existen otros problemas como la explosión o el desvanecimiento del gradiente, pero están ligados a BP como herramienta para calcularlo, por lo que se abordarán en la sección siguiente junto a la inicialización de pesos del modelo, que es la manera principal de superar estos problemas. 







\input{Plantilla_TFG_latex/Matemática/Definición}

\newpage

\part{Parte informática: enfoque clásico vs técnicas metaheurísticas}

\vspace{4cm}

\newpage



\section{Introducción}

Las redes neuronales profundas han revolucionado el campo de la inteligencia artificial, permitiendo avances significativos en varios campos como el procesamiento del lenguaje natural, reconocimiento de voz o visión por computador. Su capacidad para extraer patrones y representaciones complejas de grandes datasets y a un coste computacional muy eficiente en comparación con otras técnicas las han convertido en piedra angular de los sistemas modernos de aprendizaje automático. En el campo de la visión por computador, las Convolutional Neural Networks (ConvNets) se han erigido como la familia de modelos que consigue un rendimiento del estado del arte REFERENCIA.

Las ConvNets son un tipo de red neuronal para procesar datos en forma de grid. Se caracterizan porque tienen al menos una capa donde usan la operación de convolución en lugar de una matriz general de multiplicación. La convolución es un tipo de operación lineal que permite capturar representaciones espaciales aplicando un filtro a la entrada, detectando primero características de bajo nivel como bordes y texturas y aumentando el nivel de complejidad de la representación en las capas sucesivas. REFERENCIA

Las ResNets son una subfamilia de ConvNets que atajan el problema del desvanecimiento y explosión de gradiente. Cuantas más capas tiene una red neuronal profunda más probable es que sufra este problema, ya que se arrastran más operaciones. Las ResNet crean bloques residuales donde se crea un atajo entre el inicio y el final del bloque en el que se suma la identidad al final del bloque, provocando que el gradiente pueda fluir de manera más efectiva durante el proceso de BP.

El gradiente descendente es un algoritmo de aprendizaje que nos permite entrenar este tipo de modelos de forma eficiente, robusta, y con mucho rigor teórico. Sin embargo como ya se ha comentado en la parte anterior tiene algunas limitaciones, y estas se ven incrementadas cuantas más capas y más parámetros tiene el modelo que entrenamos. A parte de desarrollar mejoras en él con los optimizadores, se buscan nuevos algoritmos de aprendizaje que permitan evitar los problemas que presenta el gradiente descendente.

Una de estas aproximaciones son las técnicas metaheuristicas: estrategias de optimización basadas normalmente en componentes bio-inspirados y que son flexibles y adaptables a gran variedad de problemas. Ofrecen una solución cercana a la óptima en un tiempo razonable en muchos problemas cuya solución óptima es computacionalmente inalcanzable, como en problemas NP-Hard. Son técnicas iterativas que no ofrecen una garantía teórica de hallar una buena solución, pero a través de restricciones en el algoritmo se espera que lo haga.

Los más conocidos son los algoritmos evolutivos inspirados en la evolución genética. En ellos se genera una población aleatoria e iterativamente se realizan los procesos de: seleccionar los mejores individuos, recombinarlos entre ellos, mutarlos para obtener más diversidad genética, y reemplazamiento de los nuevos indiviudos en la población. Se pueden introducir modificaciones como criterios elitistas, en los que por ejemplo reemplazaríamos la población antigua sólo si fuera peor que la nueva. Los algoritmos evolutivos basados en Differential Evolution (DE) se especializan en optimización con parámetros reales y enfatizan la mutación, utilizando el operador de cruce a posteriori de ella. Son los que mejores resultado ofrecen actualmente.

Los algoritmos meméticos son una hibridación de las técnicas metaheurísticas con algoritmos de búsqueda local, que añaden el uso de información específica del problema. Combinan así la capacidad exploradora del espacio de soluciones que tienen los algoritmos evolutivos con la capacidad explotadora de la búsqueda local. El optimizador local se considera una etapa más dentro del proceso evolutivo y debe incluirse en él.



\subsection{Motivación}

El ajuste de pesos de un modelo es una de las partes más importantes en su desarrollo y por eso necesitamos de técnicas que nos ofrezcan cada vez mejores resultados a la vez que mayor eficiencia. No se trata de un problema sencillo ya que el número de parámetros de los modelos, es decir la dimensión del problema de optimización, tiene una tendencia que va rápidamente en aumento. Aunque el gradiente descente sea una estrategia muy buena hemos visto sus limitaciones, que nos incitan a intentar encontrar otras estrategias de aprendizaje. Las metaheurísticas toman cada vez un papel más protagonista en la optimización de problemas complejos y de grandes dimensiones a un bajo coste, lo que las sitúa como un posible candidato a sustituirlo.

Para la realización del presente TFG nos basaremos en el reciente paper de Daniel Molina y Francisco Herrera REF donde se analiza el papel que juegan actualmente las metaheurísticas tanto en el en entrenamiento de los modelos, como en la selección de los hiperparámetros y de la topología de la red. Nos centraremos únicamente en el primer caso. En él se realiza también un experimento práctico comparativo entre Adam, un optimizador basado en el gradiente descendente, y diferentes versiones de SHADE-ILS, una técnica metaheurística basada en DE que hace uso de búsqueda local (técnica memética) que ofrece los mejores resultados actualmente en el entrenamiento de modelos.

En dicha publicación se realiza una revisión de la literatura en lo referente a las técnicas metaheurísticas para el entrenamiento de modelos, analizando los resultados de los paper más recientes y criticando de manera general la falta de rigor metodológico en la mayoría de ellos, ya que no resulta fácil realizar una comparación totalmente objetiva entre dos ténicas tan distintas como el gradiente descendente y los algoritmos bio-inspirados. Algunas de las principales carencias en la literatura que se intentarán abordar en este TFG son las siguientes:

\begin{itemize}

\item De manera general no se suelen comparar las técnicas metaheurísticas con los métodos clásicos del gradiente descendente, sino que se comparan entre sí.

\item Falta de homogeneidad en los datasets usados, lo que no permite una comparación objetiva entre papers

\item Modelos no realistas que usan unos pocos de miles de parámetros.

\item La gran mayoría de hibridaciones entre técnicas metaheurísticas y gradiente descendente son probadas con ConvNets

\end{itemize}

Por ello, aunque la literatura sea extensa y de manera general se evidencie la superioridad del gradiente descendente, se hace necesario tanto la realización de experimentos con las mismas condiciones que en otros papers como la repetición de los mismos para obtener esa independencia, objetividad y rigor metodológico que son necesarias en el campo. Vamos a reproducir el experimento en modelos MLP y ConvNets de diferente número de parámetros (desde 2 mil hasta 1 millón), usando técnicas clásicas y metaheurísticas reconocidas por su buen funcionamiento y además probando hibridaciones entre ellas. Se intentará reproducir en la medida de lo posible las condiciones de experimentación del paper de referencia haciendo uso de algunos datasets que se proponen en la publicación y usando las mismas técnicas de entrenamiento, llegando a añadir alguna más.



\subsection{Objetivos}

El objetivo principal de este TFG es realizar una comparación experimental de las técnicas de entrenamiento de modelos clásicas basadas en gradiente descendente y las nuevas basadas en metaheurísticas, proponiendo hibridaciones para las más consolidadas. Para ello dividimos en los siguientes objetivos secundarios:


\begin{enumerate}

\item Reproducción de la experimentación del paper de referencia en el ámbito del entrenamiento de ConvNets.

\item Realización de pruebas experimentales análogas a las anteriores para el entrenamiento de MLP

\item Hibridación de técnicas metaheurísticas usadas con gradiente descendente, con sus pruebas correspondientes.

\item Análisis de resultados, en comparación con los resultados del paper cuando corresponda.

\end{enumerate}

\section{Fundamentos teóricos}
En esta sección se detallan los conceptos, familias de modelos y algoritmos necesarios para la elaboración del trabajo posterior. Se usarán los conocimientos aprendidos en las asignaturas de Aprendizaje Automático, Visión por Computador y Metaheurísticas. Además de información extraída de los artículos de publicación originales cuando corresponda, se usan las siguientes fuentes: \cite{GoodFellowBook} y \cite{stanford_231} para las secciones \ref{sec:profundo} y \ref{sec:convnets}; \cite{divedeeplearning} para la sección \ref{sec:resnets}; \cite{mhhandbook}, \cite{diffevbook},  \cite{Numerical_optimization} y \cite{stanford_231} para la sección \ref{sec:mh}; y por último \cite{divedeeplearning} y \cite{GoodFellowBook} para \ref{sec:gd}.


\subsection{Redes neuronales y aprendizaje profundo}
\label{sec:profundo}

\subsubsection{Red neuronal}

Una red neuronal es un modelo computacional inspirado en la manera en la que las neuronas se conectan en el cerebro humano procesando la información. Consiste en capas interconectadas con nodos llamados neuronas, donde cada conexión tiene un peso asociado. Cada neurona normalmente aplica una función no linear, llamada función de activación, a la suma ponderada de las entradas de la capa anterior, permitiendo al modelo aprender relaciones complejas. Este tipo de redes se denominan totalmente conectadas. Sus componentes básicos son:

\begin{itemize}
	\item Capa de entrada: recibe la información.
	
	\item Capas ocultas: son las capas intermedias, que realizan los cálculos.
	
	\item Capa de salida: produce la salida del modelo.
\end{itemize}

El ejemplo más sencillo es el Perceptrón \cite{perceptron}, una red neuronal de una sola capa oculta y una sola neurona como podemos ver en la imagen \ref{fig:Perceptron}. Es un clasificador lineal, es decir, sólo puede resolver tareas cuyos datos sean linealmente separables. En su versión original su función de activación $f$ es la función signo. Para problemas más complejos que no sean lineales necesitamos usar redes neuronales con varias capas ocultas.


\begin{figure}
    \centering
    \includegraphics[width=0.75\linewidth]{Plantilla_TFG_latex//imagenes//Inf//2.Fund/neuron_model.jpeg}
    \caption{Esquema del modelo de una neurona con tres conexiones de entrada. Obtenida de \cite{stanford_231}}
    \label{fig:Perceptron}
\end{figure}


\subsubsection{Aprendizaje profundo y redes neuronales profundas}

Las redes neuronales profundas tienen varias capas ocultas, generalmente más de dos, aunque lo normal en este tipo de modelos es que se usen alrededor de 10 o 20. La profundidad de una red viene determinada por el número de capas ocultas que tiene, y ésta permite aprender representaciones jerárquicas de los datos, lo que habilita a las redes neuronales profundas a capturar patrones más complejos en los datos en comparación con redes con menos profundidad.

El aprendizaje profundo es una subrama del aprendizaje automático que se centra en las redes neuronales profundas. Generalmente se usan modelos muy profundos con un alto número de parámetros y grandes cantidades de datos para resolver tareas complejas. Esto supone que se requiere de mucho poder computacional y de algoritmos avanzados para optimizar sus parámetros. Este tipo de modelos obtiene un gran rendimiento en tareas como el procesamiento del lenguaje natural, reconocimiento de voz o visión por computador. Los MLP son el ejemplo más clásico.





\subsubsection{Perceptrones multicapa}

Los perceptrones multicapa o \textit{Multilayer Perceptrons} (MLP) son una versión más compleja del Perceptrón, que cuenta con varias capas ocultas y varias neuronas en cada una como el ejemplo de la figura \ref{fig:NeuralNet}. Son capaces de procesar datos no linealmente separables ya que pueden aprender información más compleja. Sus capas son totalmente conectadas y la información fluye sólo hacia delante a la hora de hacer una predicción con el modelo. 

En su definición original, usan la función signo como función de activación en todas las neuronas y sólo se usan para tareas de clasificación. Sin embargo actualmente son sinónimo de redes profundas totalmente conectadas, siendo usadas con cualquier tipo de función de activación y para tareas de clasificación o regresión.

\begin{figure}
    \centering
    \includegraphics[width=0.75\linewidth]{Plantilla_TFG_latex//imagenes//Inf//2.Fund/neural_net2.jpeg}
    \caption{Red neuronal de tres capas (sin contar la capa de entrada) con dos capas ocultas de cuatro neuronas cada una y una capa de salida con una neurona. Destacar que cada neurona se conecta solo con la siguiente capa. Obtenida de \cite{stanford_231}}
    \label{fig:NeuralNet}
\end{figure}


\subsection{ConvNets}
\label{sec:convnets}

Las ConvNets o redes convolucionales son una familia de modelos de aprendizaje profundo usadas en la visión por computador. Obtienen un rendimiento al nivel del estado del arte en tareas como el reconocimiento de imágenes o la detección de objetos. Se caracterizan por tener una o varias capas (al menos una) basadas en convoluciones para luego tener una o varias capas totalmente conectadas. Las primeras sirven como extractores de características que capturan propiedades espaciales de las imágenes, mientras que las segundas sirven para clasificación. Comenzaron a ganar popularidad con el modelo LeNet5 \cite{lenet5} presentado por Yann LeCun en 1998, consiguiendo superar en rendimiento al resto de técnicas hasta la fecha en el reconocimiento de dígitos manuscritos (MNIST).

\subsubsection{Operación de convolución}

La convolución es una operación matemática que expresa la relación entre la entrada, la salida y la respuesta del sistema a los impulsos. En el contexto del procesamiento de señales, la convolución combina dos señales para producir una tercera. Se define matemáticamente para funciones continuas como

$$ (f \ast g)(t) = \int_{-\infty}^{\infty} f(x)g(t-x)dx.$$

Para funciones discretas se define como

$$ (f \ast g)(t) = \sum_{x=-\infty}^{\infty} f(x)g(t-x).$$

Nos referimos a $f$ como la entrada y a $g$ como el núcleo o filtro. En el aprendizaje automático la entrada suele ser un tensor de datos y el filtro un tensor de parámetros que adaptamos con el algoritmo de aprendizaje. Ambos son de dimensión finita y asumimos que su valor es 0 en todos los puntos donde no almacenamos su valor. Por tanto en la práctica podemos implementar la sumatoria infinita como una suma finita de los elementos de un vector. Si usamos una imagen bidimensional $I$ como entrada, seguramente usaremos un filtro bidimensional $K$:

$$S(i,j) = (I \ast K)(i,j) = \sum_m \sum_n I(m,n)K(i-m,j-n).$$


Las convoluciones cumplen las siguientes propiedades:

\begin{itemize}

	\item \textbf{Conmutatividad}: $f \ast g = g \ast f$.
	
	\item \textbf{Asociatividad}: $f \ast (g \ast h ) = ( f \ast g) \ast h$.
	
	\item \textbf{Distributividad}: $f \ast (g + h) = (f \ast g) + (f \ast h)$.

\end{itemize}

La propiedad conmutativa es útil a nivel matemático pero no es demasiado práctica en la implementación de una red neuronal. Por ello muchas librerías de aprendizaje automático optan por implementar la función llamada relación cruzada en lugar de la convolución, volteando el núcleo como podemos ver en la figura\ref{fig:3.Conv}.

$$C(i,j)= (I \cdot K)(i,j) = \sum_m \sum_n I(i+m,j+n)K(m,n).$$


\begin{figure}
    \centering
    \includegraphics[width=0.5\linewidth]{Plantilla_TFG_latex//imagenes//Inf//2.Fund/Conv.png}
    \caption{Convolución 2D sin voltear el filtro (relación cruzada). Obtenida de \cite{GoodFellowBook}}
    \label{fig:3.Conv}
\end{figure}


Al igual que hacen estas librerías, llamamos a estas dos operaciones indistintamente convolución, ya que en el contexto del aprendizaje de modelos no habrá diferencia, porque el algoritmo de aprendizaje obtendrá los mismos valores para el núcleo y sólo variará su posición. Podemos considerar la convolución como una multiplicación matricial donde la matriz tiene restricciones en muchas posiciones las cuales deben tener el mismo valor.

\subsubsection{Capa Convolucional}


Las capas convolucionales son las más importantes en la arquitectura de una ConvNet. Las claves de su gran rendimiento en el campo de la visión por computador son: la conectividad local, la disposición espacial y compartir parámetros.

La conectividad local hace referencia a las conexiones de las neuronas. En entradas de alta dimensionalidad como imágenes no es práctico conectar una neurona con todas las neuronas del volumen anterior, por tanto cada neurona se conecta solo a una región local del volumen de entrada. Esto viene determinado por un hiperparámetro llamado campo receptivo, que es el tamaño del filtro que aplicamos. Mientras que las conexiones son locales en el espacio 2D (ancho y altura), siempre abarcan toda la profundidad del volumen de entrada.

Con la disposición espacial nos referimos al tamaño del volumen de salida y cómo están organizadas estas neuronas. Hay tres hiperparámetros con los que controlamos esto:

\begin{enumerate}

	\item Profundidad del volumen de salida: corresponde a la cantidad de filtros que queremos usar.
	
	\item \textit{Stride}: Indica el número de píxeles (hablando en términos de imágenes) que usamos para desplazar el filtro al realizar la convolución.
	
	\item \textit{Padding}: A veces, para mantener la dimensión de la salida es conveniente rellenar el borde de la entrada con ceros.

\end{enumerate}


Las dimensiones del volumen de salida podemos calcularlas como una función dependiente del tamaño del volumen de entrada $W$, el tamaño del filtro $F$, el \textit{stride} $S$ y el \textit{padding} $P$ que queramos aplicar. La fórmula es la siguiente:

\begin{equation}\label{eq:output}
\frac{W-F-2P}{S+1}.
\end{equation}

Esta nos dará las dimensiones en ancho y altura del volumen de salida como podemos ver en la imagen \ref{fig:stride}, y su profundidad vendrá totalmente determinada por el número de filtros que queramos usar.

\begin{figure}
    \centering
    \includegraphics[width=0.75\linewidth]{Plantilla_TFG_latex//imagenes//Inf//2.Fund/2.stride.jpeg}
    \caption{Ilustración de la disposición espacial. El tamaño de la entrada es $W=5$ (vector gris) y el del filtro $F=3$ (vector verde), sin usar \textit{padding} ($P=1$). En el ejemplo de la izquierda se usa \textit{stride} $S=1$, mientras que en el de la derecha se usa $S=2$, obteniendo tamaños de salida de 5 y 3, respectivamente. Estos tamaños se pueden calcular según la fórmula \ref{eq:output}. Obtenida de \cite{stanford_231}}
    \label{fig:stride}
\end{figure}


Compartir parámetros en una ConvNet nos permite reducir el número de éstos, reduciendo el coste del entrenamiento. Se basa en la suposición de que si una característica es útil en una posición espacial $(x,y)$ tambien lo será en otra cercana $(x',y')$. En un volumen $W \times H \times D$, en lugar de que cada neurona tenga su conjuntos de pesos, tenemos $D$ conjuntos de pesos, reduciendo drásticamente su número. 

\subsubsection{Capa \textit{Pooling}}

Su función es reducir progresivamente el tamaño de la representación para reducir el número de parámetros y la carga computacional en la red, además de controlar el sobreajuste. Opera independientemente a lo largo de la profundidad del volumen, usando la operación máximo. La opción más común es usar filtros $2 \times 2$ con un \textit{stride} $S= 2$ como se observa en la figura \ref{fig:pooling}. La dimensión de la profundidad permanece intacta. Existen otros tipos de \textit{pooling}, por ejemplo realizar la media entre los elementos, pero esta opción fue dejando paso a la de seleccionar el máximo ya que obtiene mejores resultados en la práctica.

\begin{figure}
    \centering
    \includegraphics[width=0.75\linewidth]{Plantilla_TFG_latex//imagenes//Inf//2.Fund/maxpool.jpeg}
    \caption{Capa de \textit{pooling} con un filtro de tamaño $2 \times 2$ y \textit{stride} 2. Obtenida de \cite{stanford_231}}
    \label{fig:pooling}
\end{figure}

\subsubsection{Capa \textit{Batch Normalization}}

\textit{Batch Normalization} (BatchNorm) \cite{batchnorm} es un método de reparametrización adaptativa motivado por la dificultad de entrenar modelos muy profundos. En una capa de BatchNorm se estandariza la entrada a través de escalarla y trasladarla, lo que ayuda a estabilizar y acelerar el entrenamiento. Para cada mini-batch, se realiza el siguiente proceso:

\begin{enumerate}

	\item Se calcula la media $\nu_{B}= \frac{1}{m} \sum_{i=1}^{m}x_i$ y la varianza $\sigma^{2}_B=\frac{1}{m} \sum_{i=1}^m(x_i - \nu_B)^2$.
	
	\item Se normaliza la entrada: $\hat{x}_i = \frac{x_i - \nu_B}{\sqrt{\sigma^{2}_B + \epsilon}}$, donde $\epsilon$ es una constante pequeña para evitar la división por 0.
	
	\item Se escala y se traslada la entrada normalizada: $y_i=\gamma \hat{x}_i + \beta$, donde $\gamma$ y $\beta$ son parámetros aprendibles por el modelo.
\end{enumerate}

Normalizando la entrada conseguimos hacer el proceso de entrenamiento más estable e intentar evitar el problema de la exlosión o desvanecimiento del gradiente. También proporciona flexibilidad y mejora el rendimiento al reescalar y trasladar la entrada, y que esto dependa de parámetros aprendibles.

\subsubsection{Capa totalmente conectada}

Al final de las redes convolucionales lo más común es encontrarnos una o varias capas totalmente conectadas o \textit{fully connected} (FC), es decir, que cada neurona está conectada a todas las neuronas de la capa anterior, de la misma manera que ocurre en un MLP. Esta parte de la red permite clasificar las características extraídas por las capas convolucionales.


\subsection{ResNets}
\label{sec:resnets}

Las ResNets (Residual Networks) \cite{ResNets} son una familia de modelos dentro de las ConvNets. Su característica principal es que usan bloques residuales, que agrupan varias capas en los cuales se suma la identidad (la entrada al bloque) a la salida del bloque. Que las redes neuronales profundas aprendan esta función identidad previene el problema de la degradación, es decir, que el rendimiento de la red decaiga a medida que aumenta el número de capas. Esto puede surgir por varias causas como la inicialización de los pesos, la función de activación o el desvanecimiento/explosión del gradiente. 

\subsubsection{Bloques residuales}

La función identidad se aprende a través de las conexiones residuales, que conectan el inicio y el final de los bloques residuales pasando la identidad. Estas conexiones además permiten aliviar el problema del desvanecimiento de gradiente. Vamos a centrarnos en una red neuronal de manera local, como se muestra en la figura \ref{fig:resblock}. 

Si la función identidad $f(x)=x$ es el mapeo subyacente deseado, la función residual equivale a $g(x)=0$ y, por tanto, es más fácil de aprender: sólo tenemos que llevar a cero los pesos y sesgos de la última capa de pesos dentro de la línea de puntos. Con los bloques residuales, las entradas pueden propagarse más rápidamente a través de las conexiones residuales entre capas.

\begin{figure}
    \centering
    \includegraphics[width=0.75\linewidth]{Plantilla_TFG_latex//imagenes//Inf//2.Fund/resblock.png}
    \caption{En un bloque convolucional estándar (izquierda), la parte dentro de la línea de puntos debe aprender directamente la función $f(x)$. En un bloque residual (derecha), la parte dentro de la línea de puntos debe aprender la función residual $g(x)=f(x)-x$, haciendo que la función identidad $f(x)=x$ sea más fácil de aprender. Obtenida de \cite{divedeeplearning}}
    \label{fig:resblock}
\end{figure}

Para ello necesitamos que la entrada y la salida del bloque tengan el mismo tamaño. Si reducimos la dimensionalidad de la entrada o aumentamos el número de filtros entonces deberemos modificar la entrada a través de convoluciones $1 \times 1$ para que tenga el mismo tamaño que la salida, como se muestra en la figura \ref{fig:resblock1x1}.

\begin{figure}
    \centering
    \includegraphics[width=0.75\linewidth]{Plantilla_TFG_latex//imagenes//Inf//2.Fund/resblock1x1.png}
    \caption{Bloque residual donde la entrada tiene la misma dimensión que la salida (izquierda), y bloque residual donde se transforma el tamaño de la entrada a través de convoluciones $1\times 1$ para que tengan el mismo tamaño (derecha). Obtenida de \cite{divedeeplearning}}
    \label{fig:resblock1x1}
\end{figure}

\subsubsection{Convoluciones 1x1}

%https://blog.paperspace.com/network-in-network-utility-of-1-x-1-convolution-layers/

Los bloques residuales nos permiten aumentar la profundidad de la red evitando ciertos problemas asociados, pero al añadir más capas estamos aumentando considerablemente el número de parámetros. Las convoluciones con tamaño de filtro $1 \times 1$ son una herramienta poderosa para reducir el número de parámetros manteniendo la expresividad de la red. Fueron presentadas en \cite{bottleorig}. Este tipo de convoluciones se realizan antes de realizar la convolución requerida, de manera que la dividamos en dos, una con tamaño de filtro 1 y la otra con el tamaño original.

\begin{ejemplo}
	Para ver la diferencia en el número de parámetros al usar esta herramienta, calcularemos los parámetros necesarios para realizar una convolución en el caso de tener $C=256$ canales de entrada, $O=512$ canales de salida y tamaño del filtro $F=3$.
	
	Si no usamos convoluciones $1 \times 1$, tendríamos $P= F \times F \times C \times O + O = 1.180.160$ parámetros.
	
	Usándolas debemos elegir un tamaño de filtro intermedio, por ejemplo $C'=64$. Aplicamos primero la convolución $1 \times 1$: $P'_1= 1 \times 1 \times C \times C' + C'=16.448$ parámetros. A continuación realizamos la convolución con el tamaño de filtro original: $P'_2: F \times F \times C' \times O + O=295.424$. En total, sumando las dos capas tendríamos $311.872$ parámetros, unas cuatro veces menos que en el caso anterior.
\end{ejemplo}

\subsection{Política de un ciclo de Leslie} \label{sec:leslie}

En \cite{leslie1} se presenta una estrategia que modifica la tasa de aprendizaje de manera cíclica. En lugar de usar valores fijos o decrecientes, se divide el entrenamiento en ciclos, en los cuales la primera etapa se usa para aumentar la tasa de aprendizaje y la segunda para disminuirla, dentro de unos valores razonables. 

Más adelante en \cite{leslie2} el mismo autor propone una extensión de esta estrategia en la que se usa un único ciclo durante todo el entrenamiento. Es decir que la primera mitad del entrenamiento aumentamos la tasa de aprendizaje de manera lineal y en la segunda mitad la hacemos decrecer de igual manera hasta su valor original. Además se propone modificar el momento (en el caso de que se use) de manera inversa a la tasa de aprendizaje. Esto se conoce como la política de un ciclo de Leslie o \textit{One Cycle Policy}. Esta se encuentra implementada en la librería de aprendizaje automático PyTorch\footnote{\url{https://pytorch.org/}} pero no en TensorFlow\footnote{\url{https://www.tensorflow.org/}}.

Con esta estrategia se consigue en el entrenamiento una super-convergencia \cite{leslie3}, es decir se agiliza el entrenamiento un orden de magnitud más rápido que con las estrategias convencionales. Además se comprobó que usar tasas de aprendizajes altas en el valor máximo de la política de un ciclo tiene un efecto de regularización en el entrenamiento.

\subsection{Optimizadores de gradiente descendente}
\label{sec:gd}
Con el objetivo de intentar abordar los principales problemas del algoritmo de aprendizaje del gradiente descendente se han propuesto en la literatura diversas variantes, modificando la regla de actualización de los pesos. Existen optimizadores de primer y segundo orden, en función de si hacen uso sólo de la información del gradiente o también de la matriz Hessiana, respectivamente. Vamos a ver en esta sección únicamente los de primer orden, y veremos un método de segundo orden en la sección siguiente pero como parte de una metaheurística memética. Se dan tres enfoques en este ámbito: el uso de momento, tasas de aprendizaje adaptativas y la combinación de los dos anteriores. De cada uno hacemos hincapié en el que vamos a usar en el presente TFG, en orden respectivo: NAG, RMSProp y Adam.

\subsubsection{NAG}

El algoritmo del gradiente descendente es problemático en regiones de la función de error donde una dimensión tiene mucha más pendiente que otra, que son comunes alrededor de óptimos locales. En estos escenarios el algoritmo oscila y realiza poco progreso real. El momento \cite{momentumorig} es un método que acelera al algoritmo en la dirección relevante y compensa las oscilaciones, como podemos ver en la figura \ref{fig:momentum}. Esto se realiza añadiendo una fracción $\gamma$ del vector gradiente de la última iteración al vector gradiente actual.

\begin{figure}[!tbp]

  \centering
  \subfloat{\includegraphics[width=0.4\textwidth]{Plantilla_TFG_latex//imagenes//Inf//2.Fund/sgdwoutmom.png}}
  \hfill
  \subfloat{\includegraphics[width=0.4\textwidth]{Plantilla_TFG_latex//imagenes//Inf//2.Fund/sgdwmom.png}}
  \caption{Comparación entre el algoritmo de gradiente descendente sin usar el método del momento (izquierda) y usándolo (derecha). Vemos que en la figura de la derecha se diferencia la dirección del gradiente (flechas negras) y el camino seguido por el algoritmo en rojo. Imágenes obtenidas de \cite{GoodFellowBook}.}
    \label{fig:momentum}
\end{figure}

\begin{align*}	
	v_t&= \gamma v_{t-1} + \eta \nabla C(W_t) \\
	W_{t+1} &= W_t- v_t. \\
\end{align*}

El valor de $\gamma$ se sitúa normalmente alrededor de 0.9. El término del momento se incrementa en las dimensiones en las que el gradiente apunta en la misma dirección y se reduce en las que el gradiente cambia de dirección, consiguiendo una convergencia más rápida y estable. Dotamos al algoritmo de cierta inercia para reducir la brusquedad en los cambios de dirección.

El optimizador \textit{Nesterov Accelerated Gradient} (NAG) \cite{Nesterov} modifica esta idea de manera que podamos ``predecir'' a dónde nos lleva esa inercia. A la hora de calcular el gradiente de la función de coste, no lo hacemos respecto a los parámetros, sino respecto a una aproximación de los parámetros tras la iteración actual, de manera que podamos saber de forma aproximada dónde nos encontraremos después de actualizar los pesos. Se puede interpretar como una corrección del método de momento original. El valor del momento se sitúa también alrededor de 0.9.

\begin{align*}	
	v_t&= \gamma v_{t-1} + \eta \nabla C(W-v_t) \\
	W_{t+1} &= W_t - v_t. \\
\end{align*}

Mientras que usando el optimizador momento original primero calculamos el gradiente y luego realizamos un salto grande en la dirección del gradiente acumulado, NAG primero realiza un salto grande en la dirección del gradiente acumulado, mide el gradiente y despúes realiza una corrección. Esta comparación se ilustra en la figura \ref{fig:NAG}. Esta estrategia previene al algoritmo de avanzar demasiado rápido. 

\begin{figure}
    \centering
    \includegraphics[width=0.75\linewidth]{Plantilla_TFG_latex//imagenes//Inf//2.Fund/NAG.png}
    \caption{Comparación entre los métodos del momento original (vectores azules) y el momento de Nesterov. En este último, primero realizamos un salto grande en la dirección del gradiente acumulado (vector marrrón) para luego medir el gradiente de la posición al acabar el salto y realizar una corrección (vector rojo). La flecha verde indica la posición final corregida donde acaba una iteración del método NAG. Obtenida de \url{http://www.cs.toronto.edu/~tijmen/csc321/slides/lecture_slides_lec6.pdf}}
    \label{fig:resblock1x1}
\end{figure}


\subsubsection{RMSProp}

RMSProp (Root Mean Square Propagation) es un optimizador de primer orden que introduce tasas de aprendizaje adaptativas. Presentado por Geoff Hinton en su curso \cite{rmsprop}, la fórmula de actualización de los pesos es la siguiente:

\begin{align*}
	E[g]_t &= 0.9 E[g]_{t-1} + 0.1 \nabla C(W_t)^{2}\\
	W_{t+1} &= W-t - \frac{\eta}{\sqrt{E[g_t^2]_t + \epsilon}}\nabla C(W_t)
\end{align*}

Donde $E[g^2]_t$ es la media móvil en la iteración $t$, que depende solamente de la iteración anterior y del gradiente actual. Las operaciones anteriores se realizan elemento a elemento, es decir, cada peso recibe una actualización con un factor personalizado. Actualizando los pesos de esta forma, cuando el gradiente es grande en una dirección entonces el factor de ese peso será pequeño, evitando oscilaciones; mientras que si en otra dirección el gradiente es relativamente pequeño entonces el valor será grande, acelerando el proceso de entrenamiento.

Existen otros optimizadores que usan tasas de aprendizaje variables como Adagrad \cite{adagrad}, sus diferencias residen en la ventana de iteraciones y el cálculo del factor de multiplicación del peso. En este optimizador sólo se tienen en cuenta la iteración pasada y la actual, mientras que el uso de una media exponencial decreciente permite que las tasas de aprendizaje no se vuelvan demasiado pequeñas.

\subsubsection{Adam}

\textit{Adaptative Moment Estimation} (Adam) \cite{Adam} es otro método que calcula tasas de aprendizaje adaptativas para cada parámetro. Además mantiene una media exponencial decreciente de gradientes de iteraciones anteriores $m_t$ similar al momento.

\begin{align*}
	m_t&= \beta_1 m_{t-1} + (1-\beta_1)\nabla C(W) \\
	v_t&= \beta_2 v_{t-1} + (1-\beta_2)\nabla C(W)^2.
\end{align*}

$m_t$ y $v_t$ son estimaciones del momento de primer orden (media) y de segundo orden (varianza no centrada) de los gradientes, respectivamente. Son inicializados como vectores de 0, por lo que sus autores encontraron que tenían un sesgo al 0, especialmente durante las primeras iteraciones y cuando $\beta_1$ y $\beta_2$ son próximos a 1. Por tanto se calculan nuevas variables corrigiendo el sesgo:

\begin{align*}
	\hat{m}_t&=\frac{m_t}{a-\beta_1}\\
	\hat{v}_t&=\frac{v_t}{1-\beta_2}.
\end{align*}

Ahora se usan para ajustar los parámetros como hemos visto en el optimizador anterior:

$$W_{t+1} = W_t - \frac{\eta}{\sqrt{\hat{v}_t} + \epsilon} \hat{m}_t$$

Los autores proponen valores por defecto de 0.9 para $\beta_1$, 0.999 para $\beta_2$ y $10^{-8}$ para $\epsilon$.



\subsection{Metaheurísticas}
\label{sec:mh}

En su definición original las metaheurísticas son métodos que combinan técnicas de mejora local con estrategias de alto nivel para crear un proceso capaz de escapar óptimos locales y realizar una búsqueda robusta del espacio de soluciones. Aunque no hay garantía teórica de que puedan encontrar la solución óptima, su rendimiento es muy superior en algunos casos al de algoritmos exactos que requieren demasiado tiempo para completar su ejecución, especialmente en problemas complejos del mundo real. En problemas NP-Difícil por ejemplo se prioriza el uso de metaheurísticas que dan una solución cercana a la óptima en un tiempo mucho menor que algoritmos exactos.

Podemos clasificar a las metaheurísticas en dos grandes grupos en función de cómo se realiza la búsqueda por el espacio de soluciones: basadas en trayectorias y basadas en poblaciones. En las primeras el proceso de búsqueda se caracteriza por realizar una trayectoria en el espacio de búsqueda, que puede ser visto como la evolución en tiempo discreto de un sistema dinámico. En las metaheurísticas basadas en poblaciones, en cada iteración hay un conjunto de soluciones que interactúan entre sí. Nos centraremos en este último tipo ya que es el que vamos a usar.

\subsubsection{Metaheurísticas basadas en poblaciones}

Son técnicas de optimización probabilística que con frecuencia mejoran a otros métodos clásicos, e intentan imitar el mecanismo de evolución de la naturaleza a través de similitudes con la genética, como se ilustra en la imagen \ref{fig:cruce_mh}. Tiene un conjunto de soluciones denominado población, donde cada solución se llama individuo, y son generados de forma aleatoria. En cada iteración o generación, estos individuos se recombinan entre sí para intentar obtener mejores soluciones cuyo rendimiento es medido con una función objetivo. Las etapas de cada generación se pueden ver esquemáticamente en la figura \ref{fig:gen_alg}, y son:

\begin{figure}
    \centering
    \includegraphics[width=0.75\linewidth]{Plantilla_TFG_latex//imagenes//Inf//2.Fund/cruce_mh.png}
    \caption{Famosa imagen esquemática del operador de cruce en un punto para dos vectores binarios de soluciones (izquierda), comparando el proceso con la recombinación genética de cromosomas (derecha). Obtenida de \cite{holland_gen_alg}}
    \label{fig:cruce_mh}
\end{figure}

\begin{itemize}
	\item Selección: se elige una parte de la población actual, normalmente con criterios elitistas (se elige a los mejores) aunque introduciendo cierta aleatoriedad. Si el número de individuos elegidos es igual al tamaño de la población, hablamos de un modelo generacional, mientras que si es menor hablamos de un modelo estacionario. 
	\item Cruce: Los individuos seleccionados se agrupan por parejas y se combinan a través del operador de cruce. Las soluciones resultantes se denominan hijos. Operadores comunes son el cruce en un punto, el cruce en dos puntos y el cruce uniforme. 
	\item Mutación: A los hijos se les aplican cambios aleatorios en sus valores para mantener cierta diversidad genética en la población.
	\item Reemplazo: Se reemplaza la población actual con la nueva generación. Podemos reemplazarla entera o aplicar criterios elitistas, como reemplazar sólo con los mejores o reemplazar sólo si la nueva generación es mejor que la anterior.
	\item Terminación: se comprueba si se cumple la condición de parada. Criterios comunes son un número máximo de iteraciones o la convergencia de la población (falta de mejoras entre generaciones).
\end{itemize}

\begin{figure}
    \centering
    \includegraphics[width=0.75\linewidth]{Plantilla_TFG_latex//imagenes//Inf//2.Fund/alg_gen.png}
    \caption{Esquema de la ejecución de un algoritmo basado en poblaciones donde se observan las etapas de cada generación. Obtenida de \url{https://blogs.imf-formacion.com/blog/tecnologia/}}
    \label{fig:gen_alg}
\end{figure}

Los criterios elitistas hacen que la convergencia sea más rápida, pero podemos caer en una convergencia prematura por la falta de diversidad que conllevan estos criterios, de manera que nuestro algoritmo pare antes de encontrar una solución lo suficientemente buena.



\subsubsection{Differential Evolution}


Los algoritmos de DE \cite{diffev} son modelos basados en poblaciones que son particularmente efectivos para problemas de optimización continuos. Enfatizan la mutación y la realizan antes de aplicar el operador de cruce. Usan los parámetros factor de mutación $F$ y probabilidad de cruce $C_r$. Las etapas que varían, descritas en el orden que se realizan en cada generación, son las siguientes:

\begin{itemize}
	\item{Operador de mutación: Para cada solución $x_i$ de la población, se genera un vector mutante $v_i$ a partir de la siguiente expresión:
	
	$$v_i = x_{r1} + F \cdot (x_{r2} - x_{r3}).$$
	
	Donde $x_{r1}, x_{r2}$ y $x_{r3}$ son individuos seleccionados aleatoriamente con las restricciones de que $x_i  \neq x_{rj}$ y $x_{rj} \neq x_{rj'}$ con $j, j' \in \left \{ 1,2,3 \right \}$. $F$ se suele situar en la práctica ente 0 y 2.		
	}
	
	\item{Cruce: se combinan el vector solución de partida $x_i$ y el vector mutante $v_i$ para generar el vector de prueba $u_i$. Se usa el cruce binomial:
	
	$$u_{ij} = \begin{cases}
		v_{ij} & \text{si } \text{rand}_j(0,1) \leq C_r \\
		x_{ij} & \text{en otro caso}
		\end{cases} $$
		
		donde $\text{rand}_j(0,1)$ es generado aleatoriamente con una distribución uniforme entre 0 y 1 para cada componente $j$.	
	}
	
	\item{Selección: se compara el valor de la función objetivo de los vectores iniciales con el de los vectores de prueba correspondientes, y el que tenga mayor valor pasa a la generación siguiente.}

\end{itemize}

En el pseudocódigo \ref{alg:de} podemos apreciar el cambio de orden en las etapas de cada generación con respecto al esquema general de los algoritmos basados en poblaciones que veíamos en la figura \ref{fig:gen_alg}.

\begin{algorithm}
\caption{Esquema general de DE}
\label{alg:de}
	\begin{algorithmic}
		\State $t:=0$
		\State Inicializar Pob$_t$
		\State Evaluar $x \quad \forall x \in$ Pob$_t$
		\While{No se cumpla condición de parada}
			\State $t:=t+1$			
			\State Mutar Pob$_t$ para obtener Pob'
			\State Recombinar Pob' y Pob para obtener Pob'' 
			\State Evaluar Pob''			
			\State Reemplazar Pob$_t$ a partir de Pob'' y Pob$_{t-1}$
		\EndWhile
		
		
		\Return $x_i \in Pob_t : f(x_i)\leq f(x_j)$
	\end{algorithmic}
\end{algorithm}		
		
	

\subsubsection{L-BFGS-B}\label{sec:l-bfgs}


El método L-BFGS-B (\textit{Limited-memory Broyden-Fletcher-Goldfarb-Shanno with Box constraints}) \cite{L-BFGS-B} es un algoritmo Quasi-Newton, es decir, un algoritmo de optimización iterativo. Los métodos de Newton usan la matriz Hessiana de la función a optimizar para usar más información del problema y ofrecer una convergencia más rápida y estable. Para problemas complejos de  dimensionalidad elevada, calcular la Hessiana en cada paso es una tarea computacionalmente inabarcable, y los métodos de Quasi-Newton implementan una aproximación de la Hessiana para rebajar esta carga computacional. Estos métodos se diferencian entre ellos principalmente en la forma de aproximar la matriz Hessiana.

Uno de los métodos Quasi-Newton más populares es BFGS \cite{BFGS}, que usa una aproximación de la Hessiana de forma que mantiene su propiedad de definida positiva, lo que asegura una convergencia estable. Sin embargo, aunque se reduce el coste computacional, para almacenar la matriz Hessiana se requiere demasiada memoria. El método L-BFGS \cite{L-BFGS}, en lugar de guardar una matriz con $n \times n$ aproximaciones, guarda únicamente un vector de tamaño $n$ que guarda todas las aproximaciones de manera implícita. Esta variante está diseñada para problemas de alta dimensionalidad, y produce resultados similares a su versión sin la memoria limitada.

La última variante L-BFGS-B, que usamos en el presente TFG en el algoritmo SHADE-ILS, es una modificación que maneja restricciones en los valores de las variables, lo que la hace incorporar información sobre el dominio. Al usar información del gradiente y de la Hessiana, es un optimizador de segundo orden, aunque es mucho menos popular que los optimizadores de primer orden. Aunque mejora la rapidez y estabilidad de la convergencia al usar más información del problema y proporciona mejores soluciones, los problemas de aprendizaje automático a día de hoy han adquirido una dimensionalidad demasiado alta para que este tipo de métodos resulten computacionalmente asequibles, y se prefiere usar los de primer orden. Aún así vemos que se implementa dentro del algoritmo de SHADE-ILS con resultados muy satisfactorios, no usándose como método de optimización principal sino de manera complementaria al algoritmo SHADE.


\subsubsection{SHADE}

SHADE (\textit{Success-History based Adaptative Differential Evolution}) \cite{shade} es una variante avanzada del algoritmo original de DE. Consigue mejorar éste a través de guardar información histórica sobre configuraciones de los parámetros de factor de mutación ($F$) y el ratio de cruce ($CR$) que han tenido buenos resultados para poder ajustar de manera adaptativa estos parámetros y guiar el proceso de evolución, su pseudocódigo puede observarse en \ref{alg:shade}. 

Los mecanismos de cruce y de selección son los mismos que en DE, variando principalmente el mecanismo de mutación. Para generar el vector mutante $v_i$ a partir de la solución $x_i$, SHADE usa la siguiente estrategia\footnote{Esta es la estrategia original, existen más modificaciones, aunque basadas en esta propuesta}:

$$v_i = x_{r1} + F \cdot (x_p - x_i)  + F \cdot (x_{r1} - x_{r2}).$$

Donde $x_p$ es un individuo seleccionado aleatoriamente de entre los $p$ mejores de la población y que es distinto a $x_i$. También se verifica que $x_i  \neq x_{rj}$ y $x_{rj} \neq x_{rj'}$ con $j, j' \in \left \{ 1,2 \right \}$. En el algoritmo de SHADE se usa $p=1$, es decir, se elige al mejor individuo de la población.

El algoritmo inicia los parámetros de factor de mutación y ratio de cruce al valor 0.5, y los va adaptando según se va ejecutando. Para ello mantiene un archivo de memoria, que se actualiza al final de cada generación y en el que se guardan parejas de los valores de los dos parámetros que han dado lugar a mejores soluciones. Al actualizar el archivo se usa la media de Lehmer\footnote{$Lehmer(X) = \frac{\sum_{x\in X} x^2}{\sum_{x\in X} x}$}  de manera que se le da más peso a las parejas de parámetros que mejor rendimiento obtienen. Al comienzo de cada generación el algoritmo obtiene valores de $F$ y $C_r$ para cada individuo basándose en el archivo de memoria e introduciendo pequeñas modificaciones.


%REVISAR PSEUDOCODIGO
\begin{algorithm}
\caption{Algoritmo SHADE}
\label{alg:shade}
	\begin{algorithmic}
		\State $t:=0$
		\State Inicializar Pob$_t$
		\State Inicializar $A$ (archivo externo)
		\State Inicializar $M$ (memoria de parámetros)
		\State Evaluar $x \quad \forall x \in$ Pob$_t$
		\While{evals $<$ total\_evals}
			\State $t:=t+1$
			\State Seleccionar $p$ soluciones para la mutación
			\State Mutar Pob$_t$ para obtener Pob'
			\State Recombinar Pob' y Pob para obtener Pob''
			\State Evaluar Pob''
			\State Actualizar $A$ y $M$ a partir de Pob'' y Pob$_{t-1}$
			\State Reemplazar Pob$_t$ a partir de Pob'' y Pob$_{t-1}$
		\EndWhile
		
		
		\Return $x_i \in$ Pob$_t : f(x_i) \leq f(x_j) \quad \forall j$
	\end{algorithmic}
\end{algorithm}






\subsubsection{Algoritmos meméticos}

Los algoritmos meméticos son técnicas de optimización metaheurísticas basadas en el interacción entre componentes de búsqueda globales y locales, y tienen la explotación de conocimiento específico del problema como uno de sus principios. De manera general se componen principalmente de un algoritmo basado en poblaciones al cual se le ha integrado un componente de búsqueda local.

Su principal diferencia con los algoritmos evolutivos tradicionales es que usan de manera concienzuda todo conocimiento disponible acerca del problema. Esto no es algo opcional sino que es una característica fundamental de los algoritmos meméticos. Al igual que los algoritmos genéticos se inspiran en los genes y la evolución, estas estrategias se inspiran en el concepto de ``meme'', análogo al de gen pero en el contexto de la evolución cultural. Normalmente se llama ``hibridar'' a incorporar información del problema a un algoritmo de búsqueda ya existente y que no usaba esta información.

Esta característica de incorporar información del problema está respaldada por fuertes resultados teóricos. En el teorema \textit{No Free Lunch} \cite{nofreelunch} se establece que un algoritmo de búsqueda tiene un rendimiento acorde con la cantidad y calidad de información del problema que usa. Más precisamente, el teorema establece que el rendimiento de cualquier algoritmo de búsqueda es indistinguible de media de cualquier otro cuando consideramos el conjunto de todos los problemas. 

\subsubsection{SHADE-ILS}
\label{sec:shade-ils}


SHADE-ILS \cite{shadeils} es un algoritmo memético para problemas de optimización continua a gran escala. Combina la exploración del algoritmo basado en poblaciones SHADE, usado en cada generación para evolucionar a la población de soluciones, con la explotación de una búsqueda local que se aplica a la mejor solución que se tenga en esa generación. 

En la parte de búsqueda local, en el algoritmo original existe un mecanismo de elección para usar entre varias búsquedas locales, una de ellas L-BFGS-B. En el presente TFG se ha decidido usar sólo esta última, por facilidad de implementación y porque usa más información específica del problema. Por tanto no se detallará este mecanismo de elección entre búsquedas.

Las características fundamentales de esta técnica y que la diferencia con respecto a otros algoritmos meméticos son la elección de los algoritmos empleados (tanto el de búsqueda local como el basado en poblaciones) y su mecanismo de reinicio. Éste se activa cuando a lo largo de tres generaciones el rendimiento de la mejor solución no supera en más de un 5\% al de la anterior. En dicho caso, se elige una solución aleatoria de la población y se le aplica una pequeña perturbación usando una distribución normal y el resto de la población se vuelve a generar aleatoriamente. Cuando ocurre esto los parámetros adaptativos son reiniciados a los valores por defecto.

Cabe destacar que esto se realiza ya que SHADE-ILS mantiene los parámetros adaptativos del algoritmo SHADE entre generaciones. Esto tiene mucho sentido ya que al finalizar una ejecución de dicho algoritmo, sólo aplicamos búsqueda local a una solución, con lo que la gran mayoría de la población queda intacta y por tanto podemos reutilizar estos parámetros que se han ido adaptando a ella. 

SHADE-ILS mantiene un variable para guardar la mejor solución hasta ahora y otra para guardar la mejor solución desde el último reinicio, devolviendo la primera cuando finaliza el algoritmo. En la versión utilizada se ha añadido además un array para guardar el histórico de las mejores soluciones junto con su fitness correspondiente, de manera que podamos analizar y representar las mejoras que realiza el algoritmo. 


\begin{algorithm}
\caption{Algoritmo SHADE-ILS}
\label{alg:shade-ils}
	\begin{algorithmic}
		\State $t:=0$
		\State Inicializar Pob$_t$
		\State solucion\_inicial = (maximo+minimo)/2
		\State mejor\_actual = L-BFGS(solucion\_inicial)
		\State mejor = mejor\_actual
		
		\While{evals $<$ total\_evals}
			\State anterior = mejor\_actual.fitness
			\State mejor\_actual, Pob = SHADE(Pob)
			\State mejor\_actual = L-BFGS(mejor\_actual)
			\State mejora = anterior - mejor\_actual.fitness
			\If{mejor\_actual.fitness $<$ mejor.fitness}
				\State mejor = mejor\_actual
			\EndIf
			
			\If{reiniciar}
				\State Reiniciar y actualizar el mejor\_actual
			\EndIf		
		\EndWhile
		
		\Return mejor
	\end{algorithmic}
\end{algorithm}

Vemos el pseudocódigo de la implementación realizada en \ref{alg:shade-ils} y aclaramos algunas cosas que pueden no haber quedado del todo claras en favor de la claridad del pseudocódigo. Cuando generamos la población inicial, seleccionamos la peor y la mejor solución y la combinamos haciendo una media de sus elementos. A esa solución se le aplica la búsqueda local y se incluye en la población reemplazando a la peor solución. Se guardan los valores de mejora de las últimas 3 generaciones y en caso de que todas estén por debajo del 5\% se activa el mecanismo de reinicio.






\section{Estado del arte}

En esta sección el objetivo es analizar la literatura reciente y los artículos publicados sobre el entrenamiento de modelos de aprendizaje profundo, tanto a través de técnicas basadas en gradiente descendente como metaheurísticas, enfocándonos en las familias de modelos ConvNets y MLP. Para un mejor contexto, vamos a realizar una búsqueda en la base de datos de referencias bibliográficas y citas SCOPUS, con el fin de conocer el estado actual de la literatura. Para ello la primera búsqeuda que usaremos será simple para conocer de manera general sobre el entrenamiento de modelos de aprendizaje profundo:


\begin{verbatim}

TITLE-ABS-KEY ( deep  AND learning  AND training )
AND ( LIMIT-TO ( SUBJAREA ,  "COMP" ) ) 

\end{verbatim}

El total de artículos para esta búsqueda asciende a 78.378 resultados pueden apreciarse en la figura \ref{fig:scopus_deep}. En ella se aprecia un punto de inflexión en el año 2012, en el que empiezan a crecer las publicaciones anuales de manera exponencial, siendo estas prácticamente nulas previamente. Este año es en el que AlexNet consigue ganar la competición de ImageNet, con un aumento muy significativo del interés por las redes neuronales profundas a partir de entonces. 

\begin{figure}
    \centering
    \includegraphics[width=0.75\linewidth]{Plantilla_TFG_latex//imagenes//Inf//EdA/scopus_deep.png}
    \caption{Resultados de la web SCOPUS para la búsqueda TITLE-ABS-KEY ( deep  AND learning  AND training ), muestra el número de artículos por año.}
    \label{fig:scopus_deep}
\end{figure}


Ahora vamos a realizar una búsqueda en el ámbito del entrenamiento de modelos de aprendizaje profundo pero diferenciando entre las técnicas clásicas y las técnicas metaheurísticas, para ello se usan los términos definitorios además de los nombres de las técnicas que se emplean en este TFG, realizando las siguientes búsquedas:

\begin{verbatim}
	TITLE-ABS-KEY ( ( deep  AND  learning )  AND  training  AND 
	( metaheuristic  OR  metaheuristics  OR  shade  OR  shade-ils 
	OR  ( differential  AND  evolution )  OR  memetic  OR 
	genetic ) )  AND  ( LIMIT-TO ( SUBJAREA ,  "COMP" ) )

\end{verbatim}

\begin{verbatim}
	TITLE-ABS-KEY ( ( deep  AND  learning )  AND  training  AND 
	( gradient  OR  adam  OR  optimizer  OR  rmsprop  OR  nag ) )  
	AND  ( LIMIT-TO ( SUBJAREA ,  "COMP" ) ) 
\end{verbatim}

\begin{figure}[!tbp]
\label{fig:resEdA}
  \centering
  \subfloat{\includegraphics[width=0.75\textwidth]{Plantilla_TFG_latex//imagenes//Inf//EdA/scopus_mh.png}}
  \hfill
  \subfloat{\includegraphics[width=0.75\textwidth]{Plantilla_TFG_latex//imagenes//Inf//EdA/scopus_gd.png}}
  \caption{Artículos publicados por año para el entrenamiento de modelos de aprendizaje profundo con metaheurísticas (arriba) y gradiente descendente (debajo) según las búsquedas correspondientes.}
\end{figure}

Obtenemos una cantidad de 6753 artículos en lo referente al entrenamiento de modelos de aprendizaje profundo con técnicas basadas en gradiente descendente y 997 para técnicas metaheurísticas. Vemos que la diferencia entre ambas cantidades es grande, siendo la primera 7 veces mayor que la segunda. Destacamos sin embargo que la tendencia en ambos casos es muy similar, como se observa en la figura \ref{fig:resEdA} aumentando prácticamente en la misma proporción desde el año 2012.



\subsection{Gradiente descendente y optimizadores}

El gradiente descendente es el algoritmo de aprendizaje usado por defecto prácticamente en la totalidad de los modelos de aprendizaje profundo gracias a su eficiencia y buenos resultados. Los problemas que aparecen en su convergencia se buscan evitar en la práctica a través del desarrollo de modificaciones a su algoritmo llamadas optimizadores. La literatura en este sentido es extensa, existiendo multitud de optimizadores. Primero vamos a hacer una distinción clara entre optimizadores de primer y de segundo orden. 

Aunque los de segundo orden tienen mejores propiedades teóricas y proporcionan una convergencia más rápida y más estable, ya que usan más información del problema, el hecho de tener que calcular o aproximar la matriz Hessiana aumenta demasiado el poder computacional que se requiere para utilizarlos, ralentizando mucho el entrenamiento. También hay un problema de memoria, ya que para una red neuronal de 1 millón de parámetros se necesitaría almacenar una matriz de tamaño [1.000.000, 1.000.000], ocupando aproximadamente unos 3725GB de memoria RAM. Esto resulta inabarcable más aún viendo que en el top-10 modelos de clasificación de ImageNet ningún modelo baja de los mil millones de parámetros. 

Incluso si eliminamos estos inconvenientes de memoria con métodos como L-BFGS (ver sección \ref{sec:l-bfgs}, tenemos un gran inconveniente con él y es que debemos hacer el cálculo sobre todo el conjunto de entrenamiento. Estos pueden contener del orden de millones de ejemplos (ImageNet tiene más de un millón), haciendo su cálculo inasequible computacionalmente. Conseguir que este tipo de algoritmos como L-BFGS funcionen con mini-batches es más complejo que en MBGD y de hecho es un área abierta de investigación.

En la práctica no es común ver L-BFGS u otros optimizadores de segundo orden aplicados a modelos de apendizaje profundo a gran escala. En su lugar se utilizan variantes de MBGD basadas en el uso de momento y en learning rates adaptativos ya que son mucho más simples y más escalables. Existen varias opciones bastante asentadas, que forman parte de las librerías de aprendizaje automático más usadas como PyTorch y TensorFlow, que son Adam, NAG, RMSProp, AdaGrad o SGD con momento, entre otras. Para atender a las técnicas más novedosas que alcanzan un rendimiento del estado del arte, vamos a analizar la siguiente comparativa: https://akyrillidis.github.io/2020/03/05/algo.html. 

En él se hace una comparativa de varios algoritmos con learning rate adaptativo (Adam, AMSGrad, AdamW, QHAdam, Demon Adam) y con learning rate no adaptativo (SGDM, AggMo, QHM, Demon SGDM). Los modelos y datasets usados en este análisis pueden observarse en la tabla \ref{table:exp}. Una consideración muy importante que se realiza en dicha comparativa y que es bien sabida en el campo del aprendizaje automático es que el rendimiento de una técnica de entrenamiento está muy ligado al dominio específico del problema (UNIR ESTO A USAR EL MISMO OPT EN GD QUE MH), y puede ocurrir que un método que no sea de los mejores en términos generales sea el mejor en un problema específico. Pasamos ahora a describir rápidamente las técnicas más interesantes.

\begin{table}[]
\centering
\begin{tabular}{|c|c|}
\hline
\textbf{Dataset} & \textbf{Modelo}  \\ \hline
CIFAR10          & ResNet18         \\
CIFAR100         & VGG16            \\
STL10            & Wide ResNet 16-8 \\
FMINST           & CAPS             \\
PTB              & LSTM             \\
MNIST            & VAE              \\ \hline
\end{tabular}
\caption{Tabla con los datasets utilizados con sus respectivos modelos en la experimentación de la comparativa (enlace)}
\label{table:exp}
\end{table}

YellowFin \cite{yellowfin} es un optimizador con learning rate y momento adaptativo, de manera que mantiene dichos hiperparámetros en un intervalo donde el ratio de convergencia es una constante igual a la raíz del momento. AdamW es una extensión de adam en la que se utiliza penalización en los pesos del modelo de manera que exista un sesgo hacia valores más pequeños durante el entrenamiento, ya que normalmente se asocian valores grandes en los parámetros con el sobreajuste. Aunque Adam ya incorpora esto, AdamW realiza una pequeña modificación a través de desacoplar esta penalización a la actualización del gradiente, resultando en un impacto notable. QHADAM, DEMON ADAM, DEMON MOMENTUM, QHM, AGGMO.


Como conclusión, y atendiendo siempre al dominio específico del problema, se tiene que YellowFin es la mejor opción en caso de no disponer de recursos para ajustar los hiperparámetros, ya que adapta el momento y el learning rate a lo largo del aprendizaje. Si se dispone de recursos pero no demasiados, lo mejor son algoritmos de learning rate adaptativo de manera que sólo se tenga que ajustar el valor del momento, en concreto destacan AdamW, QHAdam y Demon Adam. En cambio si se quiere obtener el mejor rendimiento a toda costa, invirtiendo muchos recursos en el ajuste de parámteros, usar MBGD con momento es la mejor opción, aunque sea un método más clásico. En concreto se recomienda su uso con Demon.


\subsection{Metaheurísticas en el entrenamiento de modelos}

Aún con el uso de optimizadores, hay ciertos inconvenientes en el entrenamiento que están provocados por el uso de BP como método de cálculo del gradiente, o directamente al uso del gradiente y no a cómo se usa. Los más comunes son el desvanecimiento y explosión del gradiente y la tendencia a quedarse atascado en mínimos locales. Las técnicas metaheurísticas son una gran alternativa ya que sus operadores de búsqueda no dependen de BP evitando así sus problemas. 

Uno de los enfoques de aplicación de estas técnicas es la combinación con las técnicas clásicas, con diferentes aproximaciones. Por ejemplo en \cite{162} se usa el algoritmo \textit{Artificial Bee Colony} \cite{beesalgo} sobre un conjunto de soluciones aleatorias para generar un conjunto de soluciones al que aplicar el algoritmo de gradiente descendente. En \cite{155} se combina un algoritmo genético con el gradiente descendente, de manera que las nuevas soluciones son generadas con el operador de búsqueda del primero pero son evaluadas tras realizar varias épocas con el segundo. De manera similar en \cite{163} se usa la técnica metaheurística \textit{Particle Swarm Optimization} \cite{pso} para entrenar los parámetros de la última capa de una ConvNet, mientras que el resto se entrenan a través del algoritmo de gradiente descendente. La comparación arroja que la hibridación de ambas técnicas alcanza mejores resultados en términos de rapidez de convergencia y de \textit{accuracy}. Prácticamente la totalidad de la literatura referente a esta estrategia está centrada en ConvNets.

El otro enfoque es entrenar el modelo usando exclusivament algoritmos bio-inspirados. En este ámbito destacan los estudios \cite{174} y \cite{176} en los que se proponen dos técnicas basadas en \textit{Simulated Annealing} \cite{siman} para entrenar los parámetros de una ConvNet, consiguiendo mejor rendimiento y mejor velocidad de convergencia que en el mismo modelo entrenado mediante el algoritmo de gradiente descendente. Al igual que ocurre con el enfoque anterior, la gran mayoría de estos estudios están centrados en ConvNets y RNNs. 

Algo importante a destacar en lo referente a la literatura de entrenamiento de modelos con técnicas metaheurísticas es la falta de rigor y de un marco común en los estudios, de manera que no pueden realizarse comparaciones objetivas entre ellos. Estos detalles, que se comentan con más detalle en la sección \ref{sec:motinfo}, ponen en evidencia la necesidad de más experimentos bajo unas condiciones similares que permitan poder sacar conclusiones objetivas entre ellos.

El rendimiento de estas técnicas todavía no es comparable al de las técnicas clásicas. Si bien es cierto que para tareas con datasets sencillos y modelos con pocos parámetros pueden mejorar al gradiente descendente en minimizar la función de pérdida para el conjunto de entrenamiento, normalmente en generalización su rendimiento lo empeora. Para modelos con muchos parámetros el rendimiento de estas técnicas está por detrás, con resultados no muy distantes pero peores de manera general. Además hay que tener en cuenta la complejidad computacional, ya que para alcanzar un rendimiento similar al gradiente descendente estas técnicas necesitan de mucho más tiempo y recursos computacionales, por lo que no resultan una alternativa viable para este tipo de tareas.




\subsubsection{SHADE-ILS}

Presentamos ahora una de las técnicas metaheurísticas que mejor resultados ofrece actualmente en el entrenamiento de modelos. En esta sección nos limitaremos a valorar sus resultados en el trabajo \cite{MHtrainingClase}, mientras que su definición y funcionamiento se presentan en la sección \ref{sec:shade-ils}. En la experimentación de dicho trabajo se atienden tres cuestiones, las tres a través de técnicas metaheurísticas: diseño de la arquitectura, optimización de hiperparámetros y entrenamiento de los parámetros de un modelo. 

Nos centraremos en la última. Se utilizan 6 datasets distintos con diferente complejidad, y en base a ésta se elige una arquitectura de modelo concreta dentro de la familia de las ConvNets, de manera que tenga buen rendimiento en su entrenamiento a través de gradiente descendente. Se utiliza el optimizador Adam para esta tarea. En el entrenamiento con SHADE-ILS se utilizan diferentes estrategias que hacen uso de la estructura de capas de los modelos de aprendizaje profundo, realizando el entrenamiento en los pesos de diversas capas, según la estrategia, mientras se mantienen congelados los demás. También se realiza el entrenamiento de todo el modelo a la vez.

Los resultados de la experimentación son claros: solo en uno de las 6 tareas uno de los modelos entrenado con SHADE-ILS minimiza más la función de pérdida que el modelo entrenado con gradiente descendente. Además, en todos los casos, el error de test es mayor. Cabe mencionar que la generalización en los modelos es bastante buena, manteniéndose estos errores en valores cercanos a los que se obtiene en el entrenamiento, y aumentando el error en proporciones similares al modelo entrenado con Adam.




\section{Experimentación y entorno de ejecución}

En esta sección se detallará el entorno de pruebas junto con las justificaciones de las elecciones realizadas a lo largo de la experimentación. Para el desarrollo del código se usa el lenguaje Python principalmente con las librerías PyTorch, FastAI, Numpy, SKlearn y Pandas; implementado y ejecutado en la plataforma Paperspace, que proporciona un IDE y un entorno de ejecución online similar a Google Colab, pero en el que podemos elegir manualmente el hardware sobre el que ejecutamos el código, de manera que la comparación de tiempos y recursos entre las distintas técnicas sea objetiva. El hardware usado es Nvidia Quadro P5000, proporcionado por la plataforma. El código puede encontrarse en: \url{https://github.com/eedduu/TFG}. 

En la tabla \ref{table:exp} encontramos un esquema de los experimentos a realizar. Dada la cantidad de modelos distintos que vamos a entrenar, se ha decidido dividir el código en un archivo por tarea, teniendo cada conjunto de datos su propio archivo \verb|conjunto_de_datos.ipynb|. Se ha elegido este formato de archivo en lugar de \verb|conjunto_de_datos.py| de manera que se puedan comprobar las salidas del proyecto fácilmente. Se ha creado un módulo de python llamado \verb|utilsTFG.py| que contiene funciones comunes al código, como métricas de error propias, herramientas para el preprocesado de datos, los modelos ConvNets, funciones para graficar resultados o los algoritmos metaheurísticos. También hay un archivo \verb|comparative.ipynb| donde se realizan comparativas a posteriori de los resultados, como por ejemplo graficar relaciones entre los rendimientos de algunas técnicas o llevar a cabo test estadísticos.





\begin{table}[]
\resizebox{\textwidth}{!}{\begin{tabular}{|l|cccc|cll|}
\hline
\textbf{Familia} & \multicolumn{4}{c|}{MLP}                                                                                                                               & \multicolumn{3}{c|}{ConvNets}                                         \\ \hline
\textbf{Modelo}  & \multicolumn{4}{c|}{1,2,5 y 11}                                                                                                                        & \multicolumn{3}{c|}{LeNet5, ResNet-15 y ResNet57}                               \\ \hline
\textbf{conjuntos de datos}           & \multicolumn{1}{c|}{BHP} & \multicolumn{1}{c|}{BCW} & \multicolumn{2}{c|}{WQ}              & \multicolumn{1}{c|}{MNIST} & \multicolumn{1}{l|}{F-MNIST} & CIFAR10-G \\ \hline
\textbf{Tarea}             & \multicolumn{1}{c|}{R}                        & \multicolumn{1}{c|}{C}            & \multicolumn{1}{c|}{R} & C & \multicolumn{3}{c|}{Clasificación de imágenes}                        \\ \hline
\end{tabular}}
\caption{Resumen de la experimentación. BHP: Boston Housing Price, BCW: Breast Cancer Winsconsin, WQ: Wine Quality. R: regresión, C: clasificación. En el caso de MLP, en la fila modelo se indica el número de capas ocultas.}
\label{table:exp}
\end{table}


\subsection{Reproducibilidad}

En todas las funciones y librerías usadas en las que intervienen generación de números aleatorios se fija el valor de su semilla a 42. Esto se hace al iniciarse el proyecto, de manera que afecte a la separación de los datos en entrenamiento, de test y a la generación de parámetros iniciales para los modelos que vamos a entrenar con gradiente descendente. Luego se vuelve a fijar la semilla para todas las librerías que corresponda para generar la población que usaremos con las técnicas metaheurísticas y se fija también de nuevo antes de iniciar cada entrenamiento, de manera que se puedan repetir los experimentos por separado.

Esto último también se realiza ya que la experimentación ha debido realizarse en ejecuciones separadas, y fijando la semilla de nuevo obtenemos el mismo estado para los generadores de números aleatorios, con lo que no tenemos problema al dividir las ejecuciones. Se han guardado además, a través de la librería \verb|pickle|, tanto las poblaciones iniciales como los modelos y tiempos obtenidos para cada tarea, de manera que estos son comprobables.







\subsection{Modelos}

Usaremos dos familias de modelos: MLP y ConvNets. Con los primeros usaremos conjuntos de datos tabulares para clasificación y regresión, y con los segundos conjuntos de datos de imágenes para la tarea de clasificación. La implementación de los MLP se ha realizado a través de la librería FastAI por simplicidad ya que ofrece lo necesario para usarlos directamente. La implementación de las ConvNets se ha realizado desde cero, observando la topología de LeNet5 y las ResNets en sus papers originales, ya que en ellas sí que se han introducido ciertos cambios que se comentan más adelante. Todos los modelos han sido entrenados desde cero.

Usaremos 4 modelos de tipo MLP, con 1,2,5 y 11 capas ocultas cada uno. El número de neuronas por capa es una potencia de 2 y con estructura piramidal incremental, es decir primero aumentando el número de neuronas por capa y luego disminuyéndolo. Estas son elecciones comunes en la literatura ya que facilitan las operaciones por su estructura (la primera) y el tratamiento de los datos (la segunda).

\begin{table}[]
\centering
\begin{tabular}{|c|c|c|}
\hline
\textbf{Capas ocultas} & \textbf{Neuronas por capa}                                                                     & \textbf{Parámetros} \\ \hline
1                      & 64                                                                                             & 2238                \\ \hline
2                      & 64, 64                                                                                         & 6462                \\ \hline
5                      & 64, 128, 256, 128, 64                                                                          & 85k                 \\ \hline
11                     & \begin{tabular}[c]{@{}c@{}}32, 64, 128, 256, 512, 1024, \\ 512, 256, 128, 64 y 32\end{tabular} & 1.4M                \\ \hline
\end{tabular}
\caption{Detalles de los modelos MLP}
\label{tab:MLPmod}
\end{table}

Antes de cada capa linal hay una capa BatchNorm1D, ya que es la implementación por defecto de FastAI y mejora el rendimiento en el entrenamiento. Los parámetros asociados a este tipo de capa y a los de la capa de salida van incluidos en el cómputo anterior. Se incluye al final del modelo una capa de \textit{SoftMax} en caso de que la tarea sea clasificación.

Para los modelos basados en convoluciones usamos LeNet5 y dos ResNets, con 15 y 57 capas. En el primero sustituimos las funciones de activación por ReLU, ya que en la literatura posterior a la presentación del modelo se han demostrado superiores a las sigmoides y la tangente hiperbólica. También se han sustituido las capas de \textit{AveragePool} por \textit{MaxPool} y añadido capas de BatchNorm por los mismos motivos. En la tabla \ref{table:lenet5} se muestra la topología de este modelo, obviando las capas de \textit{Flatten} y de \textit{SoftMax}. Tiene un total de 62 mil parámetros.



\begin{table}[]
\centering
\begin{tabular}{|c|c|c|c|}
\hline
\multirow{2}{*}{\textbf{Capa}} & \multirow{2}{*}{\textbf{Dimensión}} & \multirow{2}{*}{\textbf{Kernel}} & \multirow{2}{*}{\textbf{Canales}} \\
                               &                                     &                                  &                                   \\ \hline
Convolución                    & 28x28                               & 5x5                              & 6                                 \\ \hline
BatchNorm2D                    & 28x28                               & -                                & -                                 \\ \hline
ReLU                           & 28x28                               & -                                & -                                 \\ \hline
Max Pool                       & 14x14                               & 2x2, stride 2                    & -                                 \\ \hline
Convolución                    & 10x10                               & 5x5                              & 16                                \\ \hline
BatchNorm2D                    & 10x10                               & -                                & -                                 \\ \hline
ReLU                           & 10x10                               & -                                & -                                 \\ \hline
Max Pooling                    & 5x5                                 & 2x2                              & -                                 \\ \hline
Lineal                         & 120                                 & -                                & -                                 \\ \hline
BatchNorm1D                    & 120                                 & -                                & -                                 \\ \hline
ReLU                           & 120                                 & -                                & -                                 \\ \hline
Lineal                         & 84                                  & -                                & -                                 \\ \hline
BatchNorm1D                    & 84                                  & -                                & -                                 \\ \hline
ReLU                           & 84                                  & -                                & -                                 \\ \hline
Lineal                         & num\_classes                        & -                                & -                                 \\ \hline
\end{tabular}
\caption{Topología de LeNet5 para imágenes 32x32 con un canal de entrada. Las columnas dimensión y canales hacen referencia a la salida de la capa.}
\label{table:lenet5}
\end{table}


Se han diseñado dos modelos de ResNet, uno con 15 capas y otro con 57. Los bloques convolucionales agrupan 3 capas de convolución con sus respectivas capas BatchNorm, y se usan convoluciones 1x1 para hacer cuello de botella, reduciendo así el número de parámetros sin perder expresividad de la red. Se sigue el diseño usual de esta familia de modelos, por ejemplo agrupando más bloques convolucionales en mitad de la red, con una convolución previa a los bloques convolucionales y usando solo una capa lineal. El modelo ResNet57 que se implementa tiene un total de 1.3M de parámetros, mientras que ResNet15 tiene 500 mil. Sus topologías pueden observarse en las tablas \ref{table:resnet57} y \ref{table:resnet15} respectivamente. 

\begin{table}[]
\begin{tabular}{|c|c|c|c|}
\hline
\multirow{2}{*}{\textbf{Capa}} & \multirow{2}{*}{\textbf{Dimensión}} & \multirow{2}{*}{\textbf{Kernel/Stride}} & \multirow{2}{*}{\textbf{Canales}} \\
                               &                                            &                                         &                                          \\ \hline
Convolución                    & 26x26                                      & 7x7                                     & 64                                       \\ \hline
BatchNorm2d                    & 26x26                                      & -                                       & -                                        \\ \hline
ReLU                           & 26x26                                      & -                                       & -                                        \\ \hline
MaxPool2d                      & 13x13                                      & 2x2, stride 2, padding 1                & -                                        \\ \hline
BottleneckBlock x3             & 13x13                                      & 1x1, 3x3, 1x1                           & 64                                       \\ \hline
BottleneckBlock x4             & 7x7                                        & 1x1, 3x3, 1x1, stride 2                 & 128                                      \\ \hline
BottleneckBlock x4             & 4x4                                        & 1x1, 3x3, 1x1, stride 2                 & 256                                      \\ \hline
BottleneckBlock x3             & 2x2                                        & 1x1, 3x3, 1x1, stride 2                 & 512                                      \\ \hline
AdaptiveAvgPool2d              & 512                                        & -                                       & -                                        \\ \hline
BatchNorm1d                    & 512                                        & -                                       & -                                        \\ \hline
Dropout                        & 512                                        & -                                       & -                                        \\ \hline
Lineal                         & num\_classes                               & -                                       & -                                        \\ \hline
\end{tabular}
\caption{Topología de ResNet57 para imágenes 32x32 con un canal de entrada. Las columnas dimensión y canales hacen referencia a la salida de la capa.}
\label{table:resnet57}
\end{table}


\begin{table}[]
\begin{tabular}{|c|c|c|c|}
\hline
\multirow{2}{*}{\textbf{Capa}} & \multirow{2}{*}{\textbf{Dimensión}} & \multirow{2}{*}{\textbf{Kernel/Stride}} & \multirow{2}{*}{\textbf{Canales}} \\
                               &                                     &                                         &                                   \\ \hline
Convolución                    & 26x26                               & 7x7                                     & 64                                \\ \hline
BatchNorm2d                    & 26x26                               & -                                       & -                                 \\ \hline
ReLU                           & 26x26                               & -                                       & -                                 \\ \hline
MaxPool2d                      & 13x13                               & 2x2, stride 2, padding 1                & -                                 \\ \hline
BottleneckBlock x1             & 13x13                               & 1x1, 3x3, 1x1                           & 64                                \\ \hline
BottleneckBlock x1             & 7x7                                 & 1x1, 3x3, 1x1, stride 2                 & 128                               \\ \hline
BottleneckBlock x1             & 4x4                                 & 1x1, 3x3, 1x1, stride 2                 & 256                               \\ \hline
BottleneckBlock x1             & 2x2                                 & 1x1, 3x3, 1x1, stride 2                 & 512                               \\ \hline
AdaptiveAvgPool2d              & 512                                 & -                                       & -                                 \\ \hline
BatchNorm1d                    & 512                                 & -                                       & -                                 \\ \hline
Dropout                        & 512                                 & -                                       & -                                 \\ \hline
Lineal                         & num\_classes                        & -                                       & -                                 \\ \hline
\end{tabular}
\caption{Topología de ResNet15 para imágenes 32x32 con un canal de entrada. Las columnas dimensión y canales hacen referencia a la salida de la capa.}
\label{table:resnet15}
\end{table}


Estos modelos se han elegido con el objetivo de tener una gama experimental amplia en base al número de parámetros, de manera que nos permita responder de manera adecuada al punto 2 de las cuestiones \textit{P1} y \textit{P2} que nos planteábamos en \ref{sec:objinf}. 


\subsection{Conjuntos de datos} \label{sec:conjuntos_de_datos}

Hemos elegido 3 conjuntos de datos de clasificación de imágenes usados en \cite{MHtrainingClase}, de manera que los experimentos sean comparables. Para las tareas con MLP hemos elegido tres conjuntos de datos tabulares, aunque uno de ellos (WQ) es usado para dos tareas distintas: regresión y clasificación, teniendo por tanto cuatro tareas. Se han elegido estos conjuntos de datos en base a su número de citas, su tarea y la adecuación de sus características a la batería experimental.

La intención con la elección de estos conjuntos de datos es poder responder a las cuestiones:

\begin{itemize}

\item \textit{P1,P2}. Los diferentes tamaños de los conjuntos de datos seleccionados (desde 500 hasta 15 mil), y las diferentes complejidades de las tareas asociadas (desde muy fáciles hasta complejidad media-alta) nos hacen poder responder a los puntos 1 y 3 de los análisis del rendimiento y complejidad computacional.

\item \textit{P3}. Para tareas con MLP, se han elegido dos tareas de clasificación y dos de regresión.

\end{itemize} 







\subsubsection{Tabulares}

\begin{table}[]
\centering
\begin{tabular}{|c|c|c|c|}
\hline
\textbf{Conjunto de datos}            & \textbf{BCW} & \textbf{BHP} & \textbf{WQ} \\ \hline
\textbf{Tarea}              & C            & R            & C y R       \\ \cline{1-1}
\textbf{Nº instancias}      & 569          & 506          & 1143        \\ \cline{1-1}
\textbf{Nº características} & 30           & 13           & 11          \\ \cline{1-1}
\textbf{Objetivo}           & Diagnosis    & MEDV         & Quality     \\ \cline{1-1}
\textbf{Ratio balance}      & 1.68         & -            & 80.5        \\ \cline{1-1}
\textbf{Faltan valores}     & No           & Si           & No          \\ \cline{1-1}
\textbf{Complejidad}        & Media-baja   & Media-baja        & Media-alta  \\ \hline
\end{tabular}
\caption{Información sobre los conjuntos de datos tabulares. El ratio de balance se calcula dividiendo el número de instancias de la clase más usual entre la menos usual. BCW: Breast Cancer Winsconsin, BHP: Boston Housing Price, WQ: Wine Quality. C: clasificación, R: regresión.}
\label{tab:dat_tab}
\end{table}
En la tabla \ref{tab:dat_tab} podemos ver un resumen de estos conjuntos de datos. Vamos a describirlos un poco más en profundidad:

\begin{itemize}

\item BCW \footnote{\url{https://www.kaggle.com/conjuntos de datos/uciml/breast-cancer-wisconsin-data}}: las características se calculan a partir de una imagen digitalizada de un aspirado con aguja fina de una masa mamaria. Describen las características de los núcleos celulares presentes en la imagen. Se extraen diez características como el radio, perímetro, área, etc. y de cada una de ellas se calcula la media, la desviación estándar y la peor, resultando en las 30 características finales. El objetivo a clasificar es binario, el diagnóstico puede resultar beningno o maligno. 

\item BHP\footnote{\url{https://www.kaggle.com/conjuntos de datos/altavish/boston-housing-conjunto de datos}}: se obtiene a partir de datos sobre el mercado de la vivienda en Boston. Sus características describen varios factores como los impuestos sobre cada vivienda o la tasa de criminalidad en el barrio. El objetivo a predecir es MEDV (\textit{Median Value}), es decir el valor mediano de las casas habitadas en escala de mil dólares.

\item WQ \footnote{\url{https://www.kaggle.com/conjuntos de datos/yasserh/wine-quality-conjunto de datos}}: describe varias características del vino en base a tests fisico-químicos como la densidad, el pH o los sulfatos que contiene. Debemos predecir la calidad (1-10) mediante clasificación o regresión. La mayor complejidad reside en el poco balance entre las clases a predecir.

\end{itemize}

Para estos conjuntos de datos se ha realizado un preprocesado de los datos básico y con decisiones comunes basadas en la literatura. Se han eliminado las variables que tienen menos de un 5 o 10\% (dependiendo de la cantidad de variables del conjunto de datos) de correlación con la variable objetivo. Con las parejas de variables que tienen más de un 90\% de correlación entre sí se elimina una de las dos. Se han eliminado outliers con el método \textit{zscore} y se han escalado los datos de entrada a través de la normalización. El tamaño del batch se ha elegido mediante pruebas experimentales entre los valores 32, 64 y 128. Se divide el conjunto de datos en entrenamiento-validación-test, con un porcentaje 70-10-20. 



\subsubsection{Imágenes}

Usamos como conjuntos de datos MNIST\footnote{\url{https://yann.lecun.com/exdb/mnist/}}, F-MNIST\footnote{\url{https://www.kaggle.com/conjunto de datoss/zalando-research/fashionmnist}} y CIFAR-10\footnote{\url{https://www.kaggle.com/c/cifar-10/}}. Se reducen a 10 mil imágenes para el conjunto de entrenamiento, del cual se toman 3 mil para validación; y 5 mil para test. Nos aseguramos de que las clases sigan perfectamente balanceadas después de la reducción. Se usan las imágenes con una resolución de 32x32 y un solo canal de escala de grises, adaptando las imágenes a estas dimensiones cuando sea necesario. No se realiza preprocesamiento de datos ya que se entiende que la propia red a través de las convoluciones realiza las transformaciones necesarias. Estas elecciones se realizan, al igual que la elección del tamaño del batch, para establecer un marco común con el paper de referencia. La decisión de tomar la partición de validación del conjunto de entrenamiento corresponde principalmente a reducir el tamaño del mismo debido a las limitaciones de memoria del hardware y a la necesidad de tener un conjunto de validación (a diferencia del paper de referencia) debido a que sólo realizamos una ejecución del entrenamiento.

Vamos a conocer un poco más estos conjuntos de datos. MNIST contiene imágenes de resolución $28 \times 28$ de dígitos manuscritos (0-9) como se muestra en la figura \ref{fig:mnist}. Su complejidad es baja debido a la simplicidad de las imágenes (baja resolución y escala de grises) y la naturaleza bien definida de los dígitos, habiendo poca variabilidad en los datos. Es un \textit{benchmark} estandarizado para algoritmos de clasificación de imágenes sencillos.

\begin{figure}
    \centering
    \includegraphics[width=0.75\linewidth]{Plantilla_TFG_latex//imagenes//Inf//exp/mnist.png}
    \caption{Ejemplo de 10 imágenes aleatorias del conjunto de entrenamiento de MNIST, con sus respectivas etiquetas.}
    \label{fig:mnist}
\end{figure}

F-MNIST por su parte tiene imágenes en escala de grises de 10 tipos de ropa distintos (por ejemplo pantalones, camisetas, zapatos) como vemos en la imagen \ref{fig:fmnist}, con la misma resolución que MNIST pero una complejidad media-baja. Esta diferencia se debe principalmente a la mayor variabilidad en los objetos de ropa y sus características. Las imágenes son más complejas y tienen patrones más intrincados que los dígitos.

\begin{figure}
    \centering
    \includegraphics[width=0.75\linewidth]{Plantilla_TFG_latex//imagenes//Inf//exp/fmnist.png}
    \caption{Ejemplo de 10 imágenes aleatorias del conjunto de entrenamiento de F-MNIST, con sus respectivas etiquetas.}
    \label{fig:fmnist}
\end{figure}

Por último CIFAR-10 contiene imágenes en resolución $32 \times 32 \times 3$, aunque las convertimos a un solo canal en escala de grises como podemos ver en la figura \ref{fig:cifar10}, por lo que nos referimos a este conjunto de datos como CIFAR-10-G. Incluye 10 clases de objetos como aviones, coches, pájaros, gatos, etc. La complejidad es alta debido a la naturaleza de las imágenes, que contienen una amplia variedad de objetos con diferentes formas, texturas y fondos. Las imágenes son relativamente pequeñas en resolución para la cantidad de información contenida en ellas, haciendo más difídil distinguir pequeños detalles necesarios para una correcta clasificación. La variabilidad en los datos requiere de técnicas más avanzadas para clasificación, haciéndolo un \textit{benchmark} estandarizado para evaluar modelos de aprendizaje profundo. Reducir las imágenes a un solo canal ayuda a reducir la cantidad de parámetros del modelo y el tiempo de entrenamiento, pero para saber si afecta a la complejidad de la tarea habría que realizar un análisis específico, ya que aunque perdemos información sobre los datos no está claro que sea información relevante. Para ello, por ejemplo, los colores deberían ser consistentes dentro de una misma clase, y diferentes a los colores de las demás.

\begin{figure}
    \centering
    \includegraphics[width=0.75\linewidth]{Plantilla_TFG_latex//imagenes//Inf//exp/cifar10.png}
    \caption{Ejemplo de 10 imágenes aleatorias del conjunto de entrenamiento de CIFAR-10-G, con sus respectivas etiquetas.}
    \label{fig:cifar10}
\end{figure}


\subsection{Entrenamiento}

Usamos los tres optimizadores de primer orden de gradiente descendente mencionados anteriormente con un doble objetivo. En primer lugar así tenemos resultados más diversos con los que comparar las técnicas metaheurísticas, pudiendo observar si estos algoritmos mejoran a alguno o ninguno de los optimizadores propuestos. Como los tres optimizadores son ampliamente usados en la literatura, creemos que esta información es pertinente. Por otro lado, sabemos que el optimizador que mejores resultados consiga depende ampliamente de la tarea y del modelo, por tanto, al ejecutar las técnicas hibridadas con el gradiente descendente podemos asignar a cada modelo el optimizador que mejor rendimiento haya ofrecido en la tarea.

En las estrategias metaheurísticas elegimos usar SHADE y SHADE-ILS. El primero es uno de los algoritmos sin búsqueda local que mejores resultados ofrece en optimización de problemas continuos, mientras que el segundo es un referente en la optimización de problemas continuos a gran escala y especialmente en el entrenamiento de modelos. Además proponemos dos técnicas nuevas: SHADE-GD y SHADE-ILS-GD, que resultan de la hibridación de las anteriores con el gradiente descendente. Estas decisiones nos permiten responder directamente a la cuestión \textit{P4}, mientras que la comparativa entre estos dos enfoques es necesaria para responder al resto.

Debido a la sensibilidad del entrenamiento a los parámetros iniciales, se han usado los mismos pesos iniciales para los entrenamientos con distintos optimizadores de un mismo modelo. De manera similar, se usa la misma población inicial de soluciones para un mismo modelo en cada entrenamiento con técnicas metaheurísticas. Se ha usado la inicialización de pesos Glorot, al igual que en el paper de referencia. Como diferencia respecto a dicha publicación, no usamos validación cruzada por los excesivos recursos computacionales que supondría, con lo que dividimos los datos en entrenamiento-validación-test tal como se indica en \ref{sec:conjuntos_de_datos}.

Para las tareas de regresión se ha usado el error cuadrático medio como función de coste. Es ampliamente usada en la literatura y aunque es sensible a los valores extremos, como tenemos preprocesamiento de datos vemos reducido el efecto. Se ha usado también la métrica R$^2$ para medir la explicación de la varianza con respecto a la media como predicción, para tener un criterio objetivo de comparación ya que en las dos tareas de regresión la escala del objetivo es distinta. Para las tareas de clasificación se ha usado la entropía cruzada como función de error y \textit{accuracy} como métrica, opciones ampliamente usadas en la literatura. En los conjuntos de datos tabulares se ha usado \textit{BalancedAccuracy} en lugar de \textit{accuracy}, ya que las clases no están balanceadas, y se conoce que en estos casos la segunda no es una métrica representativa, de hecho se hace especial mención a esto en \cite{MHtrainingClase}. En los conjuntos de datos de imágenes las clases están perfectamente balanceadas por lo que se usa \textit{accuracy}, aunque en este caso su versión balanceada coincidiría con la normal.

Para la elección de hiperparámetros usamos los elegidos en \cite{MHtrainingClase}, en los casos en que no podamos basarnos en el paper, usaremos los valores por defecto de PyTorch y los propuestos en los papers originales, en ese orden. Estos valores predeterminados de PyTorch, aunque no conseguirán el mejor rendimiento posible, están optimizados para funcionar bien en una variedad muy amplia de situaciones. Además el propósito es una comparación objetiva entre las técnicas de entrenamiento, no obtener el máximo rendimiento de cada una de ellas. Debido al gran número de modelos que entrenamos, no podríamos invertir el tiempo necesario para ajustar correctamente todos los hiperparámetros, en especial los referentes a los algoritmos metaheurísticos. Podemos ver los valores usados en la tabla \ref{tab:params}.

% Please add the following required packages to your document preamble:
% \usepackage{multirow}
\begin{table}[]
\centering
\begin{tabular}{cccc}
\multicolumn{2}{c}{}                                          & \textbf{Parámetro} & \textbf{Valor} \\ \hline
\multicolumn{2}{c}{\multirow{2}{*}{\textbf{Comunes}}}         & Pesos iniciales    & Glorot         \\
\multicolumn{2}{c}{}                                          & Épocas             & 20             \\ \hline
\multirow{4}{*}{\textbf{GD}} & \multirow{2}{*}{\textbf{ADAM}} & $\beta_1$          & 0.9            \\
                             &                                & $\beta_2$          & 0.999          \\
                             & \textbf{RMSPROP}               & $\alpha$           & 0.99           \\
                             & \textbf{NAG}                   & mom                & 0.9            \\ \hline
\multicolumn{2}{c}{\multirow{6}{*}{\textbf{MH}}}              & N$_{pob}$          & 10             \\
\multicolumn{2}{c}{}                                          & Max\_evals         & 4200           \\
\multicolumn{2}{c}{}                                          & Evals$_{SHADE}$    & 200            \\
\multicolumn{2}{c}{}                                          & Evals$_{LS}$       & 10             \\
\multicolumn{2}{c}{}                                          & Reinicio           & 3              \\
\multicolumn{2}{c}{}                                          & \% mejora          & 5              \\ \hline
\end{tabular}
\caption{}
\label{tab:params}
\end{table}

\subsubsection{Gradiente descendente}

El criterio para elegir los tres optimizadores ha sido el siguiente: en primer lugar decidimos incorporar tres técnicas basadas en gradiente descendente para tener diversidad en los resultados, como se expone arriba, y en concreto ese es el número de estrategias distintas en los que se dividen a granden rasgos los optimizadores de primer orden (ver sección \ref{sec:gd}). Para cada enfoque distinto, seleccionamos el optimizador en función del número de citas de su publicación, su uso en la literatura y su estandarización en librerías de aprendizaje automático.

Entrenamos usando la política de un ciclo de Leslie (ver sección \ref{sec:leslie}) para alcanzar una convergencia más rápida. Para elegir el valor máximo de la tasa de aprendizaje usamos la función \verb|lr_find()| de FastAI, opción usada ampliamente en la literatura. Hay que destacar que dos de los tres optimizadores que usamos tienen tasas de aprendizaje adaptativas, por lo que la elección de la tasa de aprendizaje es notablemente menos influyente en ellos. 

Usamos 20 épocas para el entrenamiento de los modelos con estos optimizadores, valor obtenido del paper comentado y comprobado experimentalmente que permite la convergencia en todos los entrenamientos. Durante el entrenamiento, guardamos los parámetros del mejor modelo en términos de error de validación, que será el que usemos para calcular el error de generalización y realizar las comparativas.





\subsubsection{Metaheurísticas}

Cuando entrenamos los modelos usando técnicas metaheurísticas tenemos que asignar muchos más recursos al entrenamiento si queremos alcanzar unos resultados parecidos. Una época ocurre cada vez que evaluamos el conjunto de entrenamiento entero con la función de pérdida. Utilizando 20 épocas para el entrenamiento de nuestros modelos no ocurren apenas mejoras, obteniendo un resultado equiparable al que conseguimos con las inicializaciones aleatorias de pesos, ya que en el algoritmo SHADE en cada generación tenemos que evaluar el modelo sobre el conjunto de entrenamiento un total de $N_{pob}$ veces. Con los valores usados el algoritmo solo ejecutaría dos generaciones, una cifra insignificante en este aspecto.

Para que el entrenamiento pueda ser comparable y se siga un criterio claro a la hora de asignar recursos, vamos a redefinir el concepto de época para el entrenamiento con estos algoritmos. Siguiendo el criterio establecido en \cite{MHtrainingClase} y usando como referencia el algoritmo SHADE-ILS, vamos a establecer que una época realiza un total de 210 evaluaciones sobre el conjunto de entrenamiento. Esto surge de utilizar 200 evaluaciones para el algoritmo SHADE y 10 para el algoritmo de búsqueda local. Así, al entrenar 20 épocas, tendríamos un total de 4200 evaluaciones sobre el conjunto de entrenamiento, mientras que los optimizadores realizarían 20. En el caso de ejecutar SHADE, otorgamos esas 10 evaluaciones pertenecientes a la búsqueda local también al algoritmo poblacional.

En los algoritmos meméticos, la ejecución del gradiente descendente se realiza cada dos épocas, es decir cada 420 evaluaciones. Por tanto aunque contamos las evaluaciones realizadas por el optimizador que corresponda, a efectos prácticos no restarían ejecuciones ni a la búsqueda local ni al algoritmo generacional ya que la comprobación de que se ha superado el número máximo de evaluaciones se realiza siempre al final de la generación.


Durante el entrenamiento con las técnicas metaheurísticas se evalúan los modelos únicamente sobre el conjunto de entrenamiento, guardando un array con el mejor modelo hasta esa generación y su correspondiente error. Luego, se evalúa cada modelo del array sobre el conjunto de validación y se selecciona el que menor error tenga sobre él de cara a calcular el error de generalización y comparar con el resto de técnicas. De esta manera establecemos un criterio similar al que usamos para seleccionar el mejor modelo entrenado con un optimizador basado en gradiente descendente.



En las técnicas meméticas, a la hora de entrenar una época con gradiente descendente no usamos la política de ciclos de Leslie. En primer lugar porque no está diseñada para entrenamientos tan cortos y no sería efectiva. En segundo lugar el uso de tasas de aprendizaje que aumenten hasta valores altos tiene un objetivo exploratorio del paisaje de la función de coste, elemento que ya tenemos gracias a la parte basada en poblaciones del algoritmo memético, por lo que nos interesa centrarnos más en la explotación de una buena región de dicho paisaje.


\subsection{Implementación}

En esta sección detallaremos las implementaciones propias que hemos debido realizar. Se darán nociones generales y se justificará el procedimiento, aunque para mayor información sobre el código remitimos directamente al mismo. Estas implementaciones se encuentran en el archivo \verb|utilsTFG|. Las librerías utilizadas son:

\begin{itemize}
	\item FastAI 2.7.17
	\item NBdev 2.3.31
	\item Ucimlrepo 0.0.7
	\item Torchvision 0.16.1+cu121
	\item Matplotlib 3.7.3
	\item Scikit-learn 1.3.0
	\item Scipy 1.11.2
	\item Torch 2.1.1+cu121
	\item Numpy 1.26.3
	\item Pandas 2.2.0	
	\item Pyade 1.0
\end{itemize}

\subsubsection{Funciones auxiliares}

Se presentan a continuación algunas de las funciones auxiliares que se han tenido que desarrollar para la correcta ejecución del código:

\begin{itemize}
	\item \verb|set_seed()|: fija la semilla de todas las librerías que contengan componentes aleatorios a 42. Dichas librerías son: FastAI, Random, Torch y Numpy.
	
	\item Funciones para graficar que permitan la comparación de varios modelos, ya sea entrenados a través de optimizadores o metaheurísticas.	
	
	\item Las métricas y funciones de error se implementan a partir de la librería SKlearn, de manera que puedan ser usadas directamente en FastAI, a excepción de \textit{accuracy} y la función de entropía cruzada, que usamos directamente la implementación de FastAI. 
	
	\item Funciones para reducir los conjuntos de datos de imágenes manteniendo el balance de las clases, con la opción de dividir en entrenamiento y validación; y otra para verificar el balance de las clases dentro de un conjunto dado.
	
	
\end{itemize}





\subsubsection{Metaheurísticas}

Las principales librerías de aprendizaje automático no incluyen herramientas para entrenar a través de técnicas metaherísticas ni para manejar los modelos usando estas técnicas, por lo que debemos realizar ciertas implementaciones que nos permitan integrarlas. 

El algoritmo de SHADE se implementa a través de pyade\footnote{\url{https://github.com/xKuZz/pyade/tree/master}}, una librería de Python que nos permite usar varios algoritmos basados en DE controlando sus parámetros. Se le han realizado modificaciones para mantener los parámetros adaptativos del algoritmo SHADE entre ejecuciones distintas y adaptar las estructuras de datos a las del resto del código.

Dicho algoritmo optimiza un array de valores flotantes, por lo que debemos crear las funciones necesarias para obtener los parámetros de un modelo en forma de array y luego volverlos a cargar en el modelo respetando la estructura por capas del mismo. Además se han creado funciones de coste que funcionan igual que las implementadas por FastAI, ya que están basadas en ellas, pero que manejan la estructura de datos que tenemos que usar con estos algoritmos. Dichas funciones permiten evaluar sobre el conjunto de entrenamiento, validación o test según corresponda. 

A partir del algoritmo SHADE se implementan manualmente el resto. Para la búsqueda local L-BFGS que se usa en SHADE-ILS y SHADE-ILS-GD usamos la función que ofrece la librería Scipy. Como función de error en dicha búsqueda realizamos una modificación de la función de error antes mencionada para que devuelva además el gradiente, necesario para dicho algoritmo. Para los algoritmos meméticos, a la hora de realizar el entrenamiento a través de gradiente descendente, simplemente usamos las funciones mencionadas anteriormente para cargar los pesos en un objeto \verb|learner| de FastAI y entrenar una época, para después devolver los pesos actualizados a la estructura de datos necesaria.








%\part{Parte informática}
%\setcounter{section}{0}
%\vspace{4cm}
%\section{Fundamentos previos}
%\section{Estado del arte}
%\section{Entorno de pruebas}
%PyTorch, fastai, elecciones del entrenamiento, CV, parámetros elegidos en los optimizadores, leslie.

%\paragraph{Elecciones en las pruebas}
%\begin{itemize}

 %   \item NAG, RMSProp y Adam. Uso las 3 estrategias/enfoques distintas en los optimizadores. Uso Nesterov tiene más citaciones su paper. Uso Adam frente a AdamW al tener también más citaciones y ser más comunmente usado. Uso RMSProp frente a AdaGrad ya que generalmente tiene mejor rendimiento (https://arxiv.org/pdf/1609.04747.pdf) (No hay paper para justificar citaciones)

    
  %  \item Entrenamiento con política de leslie. Convergencia más rápida y mayor generalización.

   % \item Grid search para el lr en momentum a partir de los valores devueltos por lr\_find(). En total 4: [lr, lr/10, lr\_steep, lr\_steep/10]. lr\_find() es una buena función para obtener una idea de por donde puede estar el learning rate óptimo. No lo uso en RMSProp ni Adam ya que tienen lr adaptativos y se comenta en papers que no es necesario (https://arxiv.org/pdf/1609.04747.pdf).

    %\item Para el resto de los hiperparámetros seleccionar los valores de los papers originales, en caso de que no haya usar el por defecto de PyTorch. Estos valores para los hiperparámetros se proponen tras muchas pruebas empíricas por lo que suelen ser una buena aproximación en el caso general cuando no se dispone de excesivo tiempo de tunearlos.

    %\item GS para seleccionar batch size, con Valores [64,128,256,512]. 64 es uno de los más usados pero con la política de Leslie mejora la regularización usar batch sizes grandes.

    %\item Para cada ejecución de GS uso 3 epochs. 

    %\item CV 3-fold. Con EarlyStopping de paciencia 3 y delta 0.1 o 0.01. Son valores comunes en la literatura.

    %\item Experimentación modular, orden de realización de experimentos:
     %   \begin{enumerate}
      %      \item LeNet5 con MNIST y CIFAR-10.

       %     \item Algoritmo genético. 

        %    \item ResNet18 con MNIST y CIFAR-10.

         %   \item Algoritmo memético

          %  \item Añadir más datasets

           % \item Probar variaciones en el genético/memético
        %\end{enumerate}
    
%\end{itemize}


%\subsection{Modelos}
%Describir los modelos usados
%\subsection{Datasets}
%Describir los datasets usados
%\section{Análisis de los métodos clásicos}
%Comentar los 3 optimizadores que uso, su base teórica y puntos fuertes y deficiencias de cada estrategia. 

%\paragraph{Momento}

%\paragraph{Learning Rates adaptativos}

%\paragraph{Combinación}

%\subsection{Resultados experimentales}
%Comparación del rendimiento de los 3 en las distintas pruebas. 

%\section{Nuevos enfoques y metaheurísticas}
%Comentar cuándo se empezaron a usar en ML, en qué tareas se les da mejor y en cuales se usan más. Comentar distintos tipos de metaheuristicas, enfocando en algoritmos geneticos que son los más usados y que mejor resultado dan de manera general en problemas de optimización , pero que suelen llevar a un overffiting en entrenamiento. Aun asi se usan para arquitecturas y seleccion de hiperparámetros.

%En principio los algoritmos geneticos sobreentrenan los modelos por tanto no hacer prueba empírica de un genético estándar, sino realizar trabajo teorico de ellos como de otras alternativas basadas en distintas metaheurísticas. Prueba empírica con propuesta. Poner ejemplos concretos de trabajos realizados, resultados, problemas, puntos fuertes y debiles, investigaciones futuras.


%\subsection{Propuesta} 
 %La mayoria de algoritmos geneticos se entrenan con todos los pesos a la vez sin hacer distincion por capas (Lights and Shadows in Evolutionary Deep Learning: Taxonomy, Critical Methodological Analysis, Cases of Study, Learned Lessons, Recommendations and Challenges), en este mismo paper se usa entrenamiento basado en evolutivos pero entrenando primero una capa mientras están las demás fijas. Proponer una variacion de algoritmo genetico con un nivel más de abstracción donde la representacion sea de: individuo=modelo, cromosoma=capa, gen=peso de la capa; aquí la recombinación no sería entre dos genes sino entre dos cromosomas. No he visto representaciones de este tipo en la literatura, aunque debería hacer una búsqueda más exhaustiva para asegurarme. A mi parecer esta representación se parece más al proceso de evolución (una persona tiene varios cromosomas cada uno con sus genes, y cada dos cromosomas del mismo tipo combinan sus genes). Además debido a la estructura de los modelos en FastAI y Pytorch resulta sencillo de operar con ellos de esta manera (a traves de un objeto learner). Usar para la recombinación el operador BLX-$\alpha$. Esto sería equivalente a usar una representación de un vector real pero usando el operador BLX-$\alpha$, en lugar de sobre cada valor independientemente, sobre un conjunto fijado de valores. El seleccionar esos conjuntos como las capas tiene bastante sentido a nivel semántico y mete una abstracción de alto nivel, que podría resultar beneficiosa. Al realizar modificaciones de la capa en su conjunto en lugar de a cada peso, la restricción es mayor y podría servir de regularizacion. Se estarían mejorando filtros en su conjunto en lugar de valores individuales.

%Los demás detalles serán a concretar tras la experimentación, pero en principio selección por torneo es ampliamente usado y no costoso. El número de la población suele situarse en 50, si uso un valor cercano a ese haría un modelo estacionario para ahorrar computación, podría probar con modelo generacional si uso un valor más bajo como 5 o 10. Idea para la mutación: truncamiento de valores float a un número concreto de decimales, simulando la imperfección en la copia genética, sería una mutación general (actúa sobre todo el cromosoma o individuo en lugar de sobre un gen concreto) que se iría acumulando de a poco. Previsiblemente el coste computacional no será todo lo reducido que debiera por el uso del objeto learner en lugar de estructuras de datos de bajo nivel, pero creo que si me pongo a realizar esas cosas de bajo nivel sería demasiado costoso y puede salir mal.


%Proponer otra variación basado en meméticos: cada x generaciones se realiza un entrenamiento de los modelos usando GD. De esta manera se combina la capacidad exploradora (busqueda global) de los genéticos con la capacidad explotadora (busqueda local) del gradiente descendente. 

%\subsection{Resultados experimentales}
%\section{Conclusiones}





%
%\input{capitulos/02_EspecificacionRequisitos}
%
%\input{capitulos/03_Planificacion}
%
%\input{capitulos/04_Analisis}
%
%\input{capitulos/05_Diseno}
%
%\input{capitulos/06_Implementacion}
%
%\input{capitulos/07_Pruebas}
%
%\input{capitulos/08_Conclusiones}
%
%%\chapter{Conclusiones y Trabajos Futuros}
%
%
%%\nocite{*}

\addcontentsline{toc}{section}{Bibliografía}
\newpage
\printbibliography
%\bibliographystyle{miunsrturl}
%
%\appendix
%\input{apendices/manual_usuario/manual_usuario}
%%\input{apendices/paper/paper}
%\input{glosario/entradas_glosario}
% \addcontentsline{toc}{chapter}{Glosario}
% \printglossary
%\chapter*{}
%\thispagestyle{empty}

\end{document}
